\documentclass{article}

\usepackage{graphicx,color,tabularx,amsmath,amssymb,fancyvrb,soul}
\usepackage{subcaption}
\usepackage{float}

\title{CS310 Computer Science Project \\ Final Report \\ The Creation of an Android Application to Aid a Student's Revision of Edexcel GCSE Mathematics}
\author{Ryan O'Hara \\ u1702761}

\begin{document}

\maketitle

\newpage

\tableofcontents

\newpage

\section{Abstract}
\label{section:abstract}

\section{Introduction}
\label{section:introduction}

The introduction of this report aims to complete three tasks. The first task is to provide an understanding of the problem which this project wishes to solve. The second task is to explain the features required by this project to be deemed successful. These features were laid out in the project specification document[appendix 1] which was produced at the beginning of the project. Finally, this project aims to provide the reader with an understanding of the report structure, giving brief descriptions of each section of the report to explain their purpose.

\subsection{Problem Outline}

The difficulty of GCSE Mathematics recently increased due to a change in specification by Edexcel in 2015. This change aimed to include content in the GCSE Mathematics exam which was previously a part of AS level Mathematics. This increase in difficulty of GCSE Mathematics was due to the UK Government's reform of the English and Mathematics GCSEs in 2013[20]. The relevant section from the Department of Education's statement is as such: "The new mathematics GCSE will demand deeper and broader mathematical understanding ...  The new mathematics GCSE will be more demanding". This demand for deeper and broader mathematical understanding enforced a syllabus change of GCSE Mathematics. This syllabus change was produced by Edexcel and started teaching in 2015, and examining in 2017[21]. \\

This increase in difficulty for GCSE mathematics is achieved through the introduction of more content. This content is initially covered in the classroom, but due to more content needing to be taught, less time is spent revisiting and revising content at the end of the course. This leaves students in the position of having to revise topics for the final exam at home, typically without the support of a teacher or tutor. \\

95\% of American teenagers have access to a smartphone[22] a study from 2018 stated. Also, a recent study conducted in 2017 stated that "Recent statistics show that around 69\% of the total digital media time spent by Americans was taken up by mobile devices and 87\% of the total mobile time was spent in apps in December 2016"[23]. More students than ever before have access to smartphones, and they are spending the majority of their time on these smartphones in apps. More students have access to a smartphone than have access to a desktop or laptop computer as of 2017[22]. \\

Because of this increase in the availability of smartphones, students are now turning towards technology to aid with their learning and revision for GCSE exams. This project aims to build a mobile application which can be used on the Android mobile operating system, which is the majority operating system (OS) for smartphone devices with a market share of 73\%[24]. This allows the application to be available to the majority of students studying for GCSE mathematics. \\

This project aims to create an application which focusses on testing students current knowledge instead of teaching them new information. This will help students to revise for their GCSE Mathematics exam. It could be used alongside classroom teaching, or other less conventional teaching methods which are used to replace the classroom environment, such as Khan Academy. This application aims to focus on testing skills already gained by the user from a classroom like environment, which will prepare the user for sitting their Edexcel GCSE Mathematics exams. \\

This application will have to provide users with the ability to practice skills gained from classroom learning. It must contain questions of GCSE exam standard so that students can effectively prepare for their exams using the application. It must also have the ability for users to input answers, and it must mark the answers and provide feedback to the users. A set of requirements which were created to complete these tasks are available below in the project specification subsection of this introduction. \\

\subsection{Project Specification}

This project aims to create an android application similar to the MyMaths homework functionality. This will enable student's to load exam questions in a specific topic in Edexcel GCSE Mathematics and input an answer. This answer will then be compared to the original answer, and the student will get a visual response as to whether they answered the question correctly or incorrectly. Where the student answered the question incorrectly, the correct answer should be displayed to the student so they can compare their answer with the correct one. \\

To achieve this task a set of requirements were laid out during the project specification[x]. These requirements follow the MoSCoW system for writing requirements. The MoSCoW system groups requirements into four classes, must, should, could, and won't. All of the must requirements are vital to the success of the project, they need to be completed by the end of the project. At the completion of all of the must requirements, a minimum viable product (MVP) should be created. Should requirements are features which are not essential to the completion of a MVP but are features which are expected in the final product. Could requirements are features which are not expected in the final product, but if all other requirements are completed before the deadline, the could requirements will be implemented. Won't requirements are features which will not, under any circumstances, be present in the final product. \\

The requirements present in the project specification are as follows: \\

The Android application which will be created for this project must: 

\begin{enumerate}
	\item Have a start page with several clickable buttons
	\item Be able to run on Android devices with Android versions between Android 6 and Android 10
	\item Have a list of each module found in Edexcel's GCSE Mathematics Specification[10], with each module being selectable
	\item Generate revision questions where numerical values are generated by Java's inbuilt random number generator and make sense
	\item The user has the ability to input numerical answers to the revision questions generated (requires 4)
	\item When answers are submitted the user must receive feedback relating to whether	their answer matches the correct answer for the question, and where the user’s answer is incorrect, display the correct answer. \\
	
	The Android application which will be created for this project should:
	\item Have a variety of testing difficulties, ranging from easy to hard for each module
	\item Have a way of tracking user progress over a number of days or tests per module (requires 5)
	\item Display model methods for working out the correct answer for a question where the user inputted an incorrect answer (requires 6) \\
	
	The Android application which will be created for this project could:
	\item Include resources similar to those which would be used in a classroom environment to teach the user about a topic
\end{enumerate}

Project success was also defined in the project specification. To be deemed successful, the final product produced by this project will have completed: 

\begin{itemize}
	\item all of the must requirements defined above
	\item at least one of the should requirements defined above
	\item any value of could requirements defined above
\end{itemize}

The target audience for the final product of this project is students sitting Edexcel GCSE Mathematics examinations. The average age of students sitting this exam is 16 years old. Although, students as early as year 7 (age 11 - 12) begin learning content which is examinable in the GCSE Mathematics exam. The app will be primarily designed for 15-16 year olds, but needs to be accessible to students as young as 12. This means there are a group of non-functional requirements which were not outlined in the project specification but were followed during development which are shown below: 

\begin{itemize}
	\item The language used in the application should be understandable to 12 year old school children
	\item The application should be simple and minimalistic as to not confuse new users
	\item A 12 year old who has not previously used the application but is familiar with smartphones should be able to use the application effectively with minimal prompting
\end{itemize}

These functional requirements enable anyone learning the content for the Edexcel GCSE Mathematics exams to be able to use the application and benefit from it.


\subsection{Report Structure}

This report will start by running through the background to this project. The background section will outline the current landscape of the problem, what has currently been achieved by other people. It will be followed by an explanation of the software development methodology which was used throughout the development of this application. This methodology section will explain how the work was completed from a project management point of view, for example it will outline a typical development session. \\

Section \ref{section:design} outlines the design of the application which will provide a high level overview of the application. This high level view will explain the control and data flows of the program, and any specific design choices which was made throughout the development of this project. Section \ref{section:implementation} discusses the specific implementation details of the project. For example, how the questions were created in the application. Testing will be discussed in section \ref{section:testing}. This will consist of outlining how testing was conducted, and the results of user testing. \\

Risks encountered throughout the project will be evaluated in the risks sections, which is section \ref{section:risks}. An outline of the risks will be produced, followed by how the risk was handled, which will then be followed by the outcome of the risk. Then project management will be discussed in section \ref{section:projectManagement}. This section will be a discussion on several details of the project that cannot be seen from the source code, for example the development schedule, or weekly meetings with key stakeholders. These details do not appear in the source code but impact the project greatly. The results section follows the project management section, where the requirements will be evaluated and the success of the project will be determined. \\

The conclusion will be section \ref{section:conclusion} which will summarise key findings in the report. This will be followed by the future work which could improve this project, and turn it from a proof of concept into an application which would be able to be released on the Google play store. Section \ref{section:issues} will discuss the legal, social, ethical, and environmental issues associated with this project. \\

The final section in this report will be the references section, with a section number of \ref{section:references}. Harvard referencing style is adhered to throughout this report.\\ 

\section{Background}
\label{section:background}

There are two main GCSE Mathematics revision tools in use today that are used in schools. Both of the main revision tools are explained in depth in their respective sections. There are also a number of other independent revision tools which are not actively used in schools, but are able to be downloaded and used on mobile devices. Maths Watch is an older tool which students used to use to learn content and revise for their exams. There have been small attempts to modernise Maths Watch, to limited success. Finally, this section will discuss where this application fits into the maths revision ecosystem. \\

\subsection{MyMaths}

The first tool is MyMaths[4] which is a website which aims to enrich a student's classroom learning experience. MyMaths main use case is to supplement the material covered in class by a teacher. The main feature of MyMaths is the online homework feature. This feature enables teachers to set homework to their class. This homework covers a specific topic, for example Venn diagrams, and the students can answer questions on this topic. Example homework questions on the topic of Venn diagrams can be viewed as figure \ref{figure:mymathsHomeworkQuestion1}.
\begin{figure}[H]
	\centering
	\includegraphics[width=0.9\linewidth]{./data/mymathsQuestion1.jpg}
	\caption{Venn diagram homework questions.}
	\label{figure:mymathsHomeworkQuestion1}
\end{figure}
The student's scores are then generated (as a percentage) and sent back to the teacher. \\

The values selected in the homework are unique each time a student attempts that specific homework. This prevents a student from memorizing a specific answer to pass the homework. A teacher can also see the number of attempts that a student has currently taken for the homework, and their highest score. Giving a teacher the ability to see the number of attempts that a student has at questions on a particular topic highlights where a student is struggling and can enable one on one teaching to boost the student's knowledge in that particular topic. A student can also attempt a homework which has not been set by their teacher. This allows a student to revise any topic in GCSE Mathematics on their own without a teacher. \\

MyMaths also gives a student the ability to take a lesson in any topic in GCSE Mathematics. This lesson walks through an example question in the topic, and then provides practice questions which are of a similar style to the homework questions. These questions are then marked, and where the student answers the question incorrectly, the correct answer is displayed next to the answer box. \\

MyMaths, however, is intended to be used in a web browser, and so is designed to be used on a desktop computer or a laptop. Due to the increase in smartphone adoption, more people of GCSE age have access to smartphones on a regular basis [x]. 
% find this figure somewhere
MyMaths has an iOS and Android application for tablets, but not a specific application for smartphones[5]. This prevents students who own their own smartphones but do not have access to a tablet on a regular basis from revising using MyMaths. \\

\subsection{Khan Academy}

The second main GCSE Mathematics revision tool is Khan Academy[6] which is a non-profit organization which aims to "provide a free, world-class education for anyone, anywhere"[6]. Khan Academy's main function is to teach new information to people. It does this through the use of short videos covering particular concepts. For example, there is a six minute video covering how to find a side of a triangle using the sine rule[7]. These videos aim to replace classroom teaching for those who are unable to attend school. There is a video for every topic in GCSE Mathematics and so a student could exclusively use Khan Academy to learn the content for their exams. Khan Academy also has a testing feature, where for a certain topic four questions come up which the user has to answer. This is, however, not available for all topics, and the user is limited to answering four questions at a time. \\

Khan Academy has applications in both major app stores, the Apple app store[8] and the Google play store[9]. This app runs on all major devices and has the same functionality as the website version of Khan Academy. Users can create accounts for free which keeps track of their progress and can access videos for all topics available on the website for free. \\

\subsection{Independent Revision Tools}

There are several other mathematics revision apps on both app stores, examples of which can be seen below as figure \ref{figure:appStoreApps}. These apps mostly focus on testing using some form of revision questions. These questions are either limited to multiple choice questions, or questions which have the same numbers each time they appear. This leads to two problems: random guessing for the multiple choice questions, or answer memorization for the questions which do not change their numbers. This leads to these applications not being endorsed in schools. \\

\begin{figure}[H]
	\centering
	\begin{subfigure}{0.8\textwidth}
		\frame{\includegraphics[width=0.9\linewidth]{./data/appleAppStoreApps.png}}
		\caption{Revision applications on Apple's app store}
	\end{subfigure}
	\hfill
	\begin{subfigure}{0.8\textwidth}
		\frame{\includegraphics[width=0.9\linewidth]{./data/googlePlayStoreApps.png}}
		\caption{Revision applications on the Google Play store}
	\end{subfigure}
	\caption{Revision applications on both the Apple app store and the Google play store}
	\label{figure:appStoreApps}
\end{figure}

\subsection{Maths Watch}

Maths Watch[11] is an older tool which was used for teaching purposes in schools. This is a service which provides educational videos for students to learn from. It is designed to replace both classroom teaching and as a homework tool. Mathswatch is a subscription service, which only schools can subscribe to. If a student's school is not subscribed to Maths Watch then a student cannot get access to it. \\

Maths Watch has two main methods for distributing their video content. The first method is through DVDs which are posted to schools which buy them, and the students can then get these discs from the schools with all of the videos on them. The second method is through their online portal which provides the ability for paid users to stream video lessons. The Maths Watch service also states on the subscription page of their website[12] that they provide "A bank of 1000s of interactive exam-style questions for students' independent work with instant feedback". This is similar to MyMaths where a user can submit their answers and get feedback on whether they were correct, and if they were incorrect have the correct answer displayed. \\

The Maths Watch service is used by students all throughout the UK and is an incredibly valuable tool to aid their learning. Unfortunately, Maths Watch does not provide a mobile application on either of the major app stores (Apple iOS app store, or the Google Play store). As well as this, only schools can purchase Maths Watch, so if a students' school is not a Maths Watch partner, then the student cannot get access to Maths Watch. \\

\subsection{App Development}

Research into Android application development consisted of: 

\begin{itemize}
	\item Working through Head First Android Development[25]
	\item Looking through the Android Documentation [26]
	\item Making small Android apps available on github [27]
\end{itemize}

These tools were utilised in learning the basics of Android application development before any code was written for this project. The Android documentation helped with any programming issues within the code during the development of this application. 

\section{Methodology}
\label{section:methodology}

This project was conducted using an agile methodology. The project focussed on building a base product and then adding new features incrementally with each development cycle until the final product was produced. 

%Go into detail about the scrum methodology.

\section{Design}
\label{section:design}

This application had a target audience of school children between the ages of 12-16 years old. This meant that any design would have to be simple and intuitive to prevent users from getting 'lost' within the application. The graphs for both the control flow and the data flow for this application are acyclic as is shown by the two figures below, figure [y] and figure [z] respectively. Users can only follow certain paths, and will only end up in the same place by going backwards using the inbuilt backwards button in Android, so they cannot get confused about the layout of the application. Also, the number of pages was kept to a minimum for the application to function for the same reason. This minimalst design aims to prevent confusion by keeping the application simple to understand and use. \\

\subsection{Control Flow}

Figure \ref{figure:applicationControlFlow} is a graph for the control flow of the application. Each edge is bidirectional, the user can move forward by pressing a button in the application, and backwards by hitting the back button built into the Android operating system. \\

\begin{figure}[H]
	\centering
	\frame{\includegraphics[width=0.9\linewidth]{./data/applicationControlFlow.png}}
	\caption{The control flow of the application.}
	\label{figure:applicationControlFlow}
\end{figure}

The only user choice in direction is at the start page where a user could select between the information page for the application and the modules list. The information page of the application just displays some basic information for the application. Each node in this graph represents an Android activity. An Android activity is the application equivalent to a webpage on a website. Activities are viewed as one screen in an application[15] and are intended to complete one specific task. If another task needs completing then the application should call another activity. Activities can contain fragments which can be thought of as "a modular section of an activity ... (sort of like a 'sub activity' that you can reuse in different activities)"[16]. The modules, submodules, and difficulty nodes in the control flow graph displayed above are activities which contain only list fragments. The list fragment class is a submodule of the fragment class which displays a list of items, and implements a listener which will act when the user clicks on a specific item. In the case of the modules node, the list displays all six modules which are examined in Edexcel GCSE mathematics. When the user clicks a specific module, the modules activity collects the information of the module selected, and starts the submodules activity and passes the information of the module selected onto the new activity. The submodules activity then displays the submodules for the module selected by the user. \\



% A control flow section is required in implementation to discuss how the modules node could be split into each of the six specific modules, and the submodules class could be split into each of the submodules depending on the module chosen. But it was decided to just have a modules class to make adding extra modules much more simple. 

\subsection{Data Flow}

Figure [x] is a graph of the flow of data throughout the application. When an activity starts another activity, it can pass information to it. This enables a single directional information flow throughout the application. That is why the diagram displayed as figure [x] is a directed graph. The direction of the data is denoted by an arrow, the side of the edge without the arrow is the sender of the data, and the side of the edge with an arrow is the data receiver. 

%\begin{figure}[H]
%\centering
%\includegraphics[width=0.9\linewidth]{./data/}
%\caption{}
%\label{figure:}
%\end{figure}

The data flow in this graph follows the minimalist design of this application. Data moves along a path in one direction so there is no confusion. A user can go back to a previous activity by using the inbuilt back button in Android. By doing this, the information selected further up the graph (closer to the root node) is preserved. This allows a user to utilise the back button to change the difficulty of the questions that they are answering while keeping the module and submodule information preserved. This enables simple use, where a user does not need to reselect all information to change one specific piece of data. \\

\subsection{Visual Design}

The visual design of this application is minimalist, in keeping with the rest of the app. Every page only has the necessary things displayed on it, and where applicable, only one thing displayed. Below are a few examples of this minimalist design. Figure \ref{figure:applicationStartPage} is a screenshot of the application's start page taken from the physical Google Pixel 2 device.

\begin{figure}[H]
	\centering
	\frame{\includegraphics[width=0.9\linewidth]{./data/applicationStartPage.png}}
	\caption{}
	\label{figure:applicationStartPage}
\end{figure}

This screen is made up of the least amount of distinct elements that is possible. As it is the start page of the application it has to have the application name displayed, which is the Maths Help text at the top of the screen. The bottom of the screen is dominated by two buttons, one which starts the modules activity (the modules button), and one which starts the information activity (the app information button). There is nothing else on the webpage, and this is the least amount of elements with which to follow the control and data flow graphs displayed above. \\

An example of list fragment pages is the modules activity page. This is displayed below as figure \ref{figure:applicationModulesPage}, also taken from the physical Pixel 2 testing device. 

\begin{figure}[H]
	\centering
	\frame{\includegraphics[width=0.9\linewidth]{./data/applicationModulesPage.png}}
	\caption{The modules activity page.}
	\label{figure:applicationModulesPage}
\end{figure}

% Discuss the control flow of the program, why it works the way it does etc...
% Get the control flow diagram from the presentation and discuss how pages are connected
% Make a new diagram, a more impressive one, which also shows the information transfer between the 
% separate activities.

\section{Implementation}
\label{section:implementation}

%
%

\subsection{Questions}

%
%

\section{Testing}
\label{section:testing}

The project specification[x] stated that this project would implement test driven development. The progress report[x] described that test driven development would not be used as the testing methodology for this project, and that user testing would be used throughout this project in its place. \\

The decision to change from unit testing to user testing was made because this project creates a tangible final product in the form of an Android application. Writing unit tests for an android application was deemed to an inefficient use of time, as this time could be spent improving the user experience of the application. An android application is judged by users for unquantifiable reasons, such as the "feel" of the application. This can only be tested by a person using the application, and since this determines how satisfied a user is with the application testing for the unquantifiable qualities of an application was prioritised over unit testing. \\

\subsection{Testing Methodology}

All application testing was conducted on a physical Google Pixel 2 device, as well as at least one emulated virtual device. The emulation software used is the software built into the Android Studio IDE. The emulator is called Android Virtual Device and [13]"The emulator provides almost all of the capabilities of a real Android device". The emulator provides an environment to test the application on a variety of devices with different capabilities to ensure that the application works on a wide range of Android devices. The application was tested on three virtual devices: 

\begin{itemize}
	\item HTC Nexus One (Smartphone)
	\item HTC Nexus 10 (Tablet)
	\item Google Pixel 3 (Smartphone)
\end{itemize}

These devices give a good variety in Android devices, between both tablet and smartphone, and old and new. These devices were chosen because if the application works well on all four devices, the physical Google Pixel 2 as well as the three virtual devices, then the application was likely to work on most Android devices. \\

Two types of user testing were conducted throughout this project, white box testing and black box testing. \\ 

\subsubsection{White Box Testing}

White box testing was conducted after each sprint cycle, where the developer would download the application onto their physical pixel 2 testing device as well as the range of virtual devices mentioned above, and then test that the application compiled and ran without errors. Other small tests were also conducted when testing new features. An example of this feature test would consist of the developer creating expected states of the application given specific inputs, and then the developer would follow the input sequence on all four testing devices, and ensure that the application ended up in the expected state for each device. Visual tests were also conducted, where an activity's display (which can be imagined as the application equivalent of a webpage in a website) in the application was drawn on paper, and then loaded up in the application for each of the testing devices, and the layout on the testing devices was compared with the layout drawn on paper. This testing method was used to find visual errors in the application and was conducted when a visual change was made. \\

\subsubsection{Black Box Testing}

Black box testing was conducted using users outside of the development team. These users were not shown the source code for the application, and this testing was conducted primarily on the physical Pixel 2 testing device. The user would be provided with the application and their use of it would be observed. Before any testing was conducted verbal consent was gained from the user for testing the application through observing their use case. If it was the first time the user had seen the application then they were asked to try to use it as if they had just downloaded it from their normal app store. Whereas, if a user was familiar with the application then they were asked to complete a task which utilised a new feature. They could also be asked to compare visual styles with an image of both the current and previous styles, although the user was not made aware of which style was the current application style and which was the previous application style. Where there was a change in a feature that the user had used before, they were not notified of the change, and how they dealt with the change was noted. \\

Black box testing was conducted after every major feature was added to the application, with the results being used to improve the feature. Users of different mathematical ability were asked to test the application, from colleagues who use mathematics every day in their fields of study, to those who last encountered it when they took their GCSEs. White box testing was conducted throughout development. This was because compilation tests were quick and easy to conduct, taking around one minute per test and requiring access to the physical testing device, and would display errors in the code which would then be fixed before attempting the compilation test again. \\

\section{Risks}
\label{section:risks}

This section will discuss some of the risks associated with this project which would have caused issues if they had occurred. The risks will be outlined, which will explain the risk and how it would have effected the project. Once the risk has been explained then the discussion will be around how the risk was handled in this project. Each method of handling risk will be one of the following four. The risk was either: avoided, transferred, mitigated, or accepted. When a risk is avoided, the cause of the risk is eliminated from the project. A transferred risk is one where the risk was handled by a third party, for example in the case of authentication, if single sign on is used the risk of password leakage is transferred to the third party responsible for the authentication. If a risk is mitigated then the effects of the risks are reduced either by reducing the probability of the risk occurring, or by reducing the impact of the risk on the project. An accepted risk is one where no action is taken before the risk occurs, and if the risk does occur, it is then promptly dealt with without a large impact on the project. Finally, the outcome of the risk is discussed. \\

\subsection{Project Creep}

Project creep occurs when the scope of the project slowly increases over time. This usually occurs because of one of two reasons, either a customer keeps on changing their description of the final product, or the project manager adds unnecessary features to the project. As there is no customer directly involved in the development of this project the only way project creep can occur is if the project manager adds to the initial feature set. \\

Project creep can be incredibly damaging to a project in one of two ways. The first way is that if the project does not have a predefined deadline and if features continue to be added and changed throughout the development of the project then the project may never be completed. If the project is evaluated for resources at the beginning of the project and project creep occurs, then there may not be enough resources to complete the project. The second way that project creep can be detrimental to a project is if that project has a deadline, then adding and changing features means less time is spent on the key features of the project. When a resource is constrained project creep can waste the resource and leaves less for important features. \\

For this application project creep would be changing or adding features which cause current progress to be scrapped. This is because time is the key limiting resource for this project. Any time wasted in this project limits the amount of requirements which can be completed by the end of the project. \\

Project creep was handled by outlining the features necessary for this application in the form of requirements in the project specification, as well as defining a successful project. The requirements in the project specification outline the features which should be worked on throughout this project and the definition of success provided a baseline for the features to be provided in the project. It also gives a rough idea for the order that they should be completed in, specifically for dependent requirements. This mitigates project creep by lowering the probability that it occurs. This is due to requirements being laid out in the beginning of the project, and only minorly changed throughout the project. So as long as the requirements are completed before any other feature is attempted to be added to the project then it will not creep. \\

This project did not suffer from project creep. The requirements were a key reason for this as they were converted to user stories which then dictated what was completed in a specific sprint cycle. As work was only conducted on user stories there was no ability for features which were not included in the initial requirements to be worked on, and so no time was wasted. \\

\subsection{Personal Issues}



\subsection{Poor Project Management}

\section{Project Management}
\label{section:projectManagement}

This project was a large software engineering project, especially for a team size of one. This meant that effective project management was required to keep the project going and make it a success. 

\subsection{Schedule}

A schedule was created early, and updated throughout the project to represent and handle the amount of work which had been completed. Two key versions of the schedule were displayed to various stakeholders throughout the project in key documents. These key documents were the project specification and the progress report. 

\subsubsection{Project Specification Schedule}

At the beginning of this project a schedule was created. This schedule aimed to complete all of the requirements stated in the project specification[x] by the end of the project. The schedule was created using a technique common in industry which involved some numbered cards. Each requirement was reviewed with each review consisting of analysing the relative difficulty of the requirement with respect to all the other requirements. The difficulty was quantified using the numbered cards, the cards would follow a logarithmic pattern, doubling with difficulty each time. The lowest valued card was \textsuperscript{1}/\textsubscript{4} and the largest value was 16. Any requirement with a relative difficulty value greater than 16 was then broken down into sub-requirements and these sub-requirements would then be reviewed, and broken down further if their relative value was also above 16. \\

This relative difficulty in requirements was used to create a rough schedule for the project. The schedule created for the project specification can be seen as a Gantt chart in figure \ref{figure:projectSpecGanttChart}. \\

\begin{figure}[H]
	\centering
	\includegraphics[width=\linewidth]{./data/projectSpecGanttChart.png}
	\caption{Gantt chart of the schedule produced for the project specification.}
	\label{figure:projectSpecGanttChart}
\end{figure}

This schedule was an optimistic schedule. It assumed all of the requirements would be completed by the end of the project, whereas the project specification's definition of success did not need all requirements to be completed. 

%
%
%

\subsubsection{Progress Report Schedule}

The progress report submitted on 24\textsuperscript{th} November 2019 updated this schedule, once it was realised that development was slower than initially intended through term one of this academic year (October 2019 - December 2019) due to personal reasons. The new schedule, referred to from now on as the Christmas schedule, again made the assumption that every requirement would be completed before the deadline. To mitigate against the slow development conducted through term one, the Christmas schedule increased the workload of the project throughout the Christmas period (8\textsuperscript{th} December 2019 - 8\textsuperscript{th} January 2020) and throughout term two of the project. \\

A Gantt chart created for this updated schedule can be viewed below as figure \ref{figure:progressReportGanttChart}

\begin{figure}[H]
	\centering
	\includegraphics[width=0.9\linewidth]{./data/progressReportGanttChart.png}
	\caption{Gantt chart of the updated schedule produced for the project specification.}
	\label{figure:progressReportGanttChart}
\end{figure}

%
%
%

\subsubsection{Schedule Evaluation}

The image displayed below as figure [x] is an assessment of the completion of the final requirements in respect to the project specification schedule. 

%
%
%

\subsection{Tools}

This project utilized several tools to aid in the development of this Android application. Two of the main tools used to add value to this project are outlined below, one from a development point of view and one from a project management standpoint. 

\subsubsection{Android Studio}

Android Studio was used as an IDE (Integrated Development Environment) throughout development as it comes with many useful built in features. These features are specifically designed with android app development in mind. Examples of these features are: 

\begin{itemize}
	\item Rendering engine which converts XML into how it will appear in the application itself
	\item Android phone emulator
\end{itemize}

The rendering engine enables both the ability to edit the XML tags and see the effects in real time, as well as the ability to edit directly into the rendering engine with a drag and drop feature. This enables quick prototyping for new pages through the use of the drag and drop feature whilst also allowing a much larger degree of control through the ability to directly edit the XML.\\

Use of the android phone emulator enabled quick and efficient testing cycles. When a feature was initially developed, it could be tested on a wide range of devices through the phone emulator. This add on enabled the emulation of any modern phone through the ability to download files specific to that device. The app could then be tested on an accurate test bed, seeing how the app would look and work on a mobile device. Tablets could also be emulated, which helped with ensuring the application worked on a wide variety of devices.\\

The Android Studio IDE greatly increased the efficiency of development. This was encouraged through the notion that it was only designed specifically for android development, and so could focus entirely on that. Being built upon the IntelliJ platform meant that code editing was smooth and intuitive. \\

\subsubsection{Git}

Git was used as version control for this project. This enabled access to the project from any device, anywhere, anytime. It also enabled the ability to revert back to an earlier version of the project if something went wrong. This meant that development could take place without fear of failure, as if anything went wrong the project could just be reverted to the last working state.\\

The repository for this project can be found at the link in reference [14]. This repository is public and can be accessed by anyone. This enables anyone to see the code that would be running on their machines.\\

Git also logs the time when a change is made to the code. This enabled precise tracking of requirement completion and aided project management. Git commit messages for this project were all deliberately helpful. Git commit messages were written so that they would explain the changes made to the codebase, so they could be logs of when project requirements were completed. The git commit messages for this project can be viewed and give a detailed timeline of when requirements were completed and features were added. This was incredibly useful as it gave updates of any changes made to the codebase as soon as it was available to anyone who follows the repository, and also was used to create a timeline of when requirements were completed for this final report. \\

% Move into the methodology section. Certainly the information about the notes file.
\subsection{Weekly Work????}

Weekly meetings were held with the project supervisor to keep key stakeholders in the project up to date with development. Alongside weekly meetings, access was given to the project supervisor to the  github repository early in development. This enabled the project supervisor to directly monitor completed work. \\

A document in the github repository called notes.txt is a file which was edited after every development session was completed. This file contained notes on what was intended to be completed during each development session, as well as what was actually completed. A substantial amount of time was spent at the beginning of early development sessions on remembering what had been completed throughout the previous development session, what state the code was in, and what needed to be completed in this development session. This file cut down the time spent at the beginning of development sessions looking through previous git commits, and so overall this file caused development to be more efficient. \\

% \subsection{Project Management Evaluation} [Maybe don't include this]

% Use analysis from project management to assess the overall management of the project


\section{Results}
\label{section:results}

This section will evaluate the final application produced by this project and compare it against the requirements and definition of success set out at the beginning of this report. It will also contain the analysis of the project by the author. 

\subsection{Requirements}

The final application satisfies all of the must requirements. This means that the final application can display a question for users to answer, have an input field accepting numerical values, and when an answer is submitted, compare it to the pre-calculated answer and display whether the two answers match. If the two answers are different, then the correct answer is displayed for the user to see. \\

One of the three should requirements for this project was completed outright. Requirement 7 was changed between the progress report and the project presentation. It was changed from "Have a variety of testing difficulties, ranging from easy to hard for each module" to "Have two separate testing difficulties, foundation and higher, for each module". This change was made as the Edexcel GCSE Mathematics exams are split into two categories, of foundation and higher. The foundation tier of exams range from grades 1-5 and the higher range from grades 4-9, with the higher tier covering all content in the foundation tier and build on it. As this tier system already existed, and changing to easy, medium, and hard would only confuse students, the decision to switch to Edexcel's system was made. \\

Requirements 8 and 9 were not completed but are in progress. These requirements have been started and prototypes for each requirements have been made. Model working out has been created for three different questions in the algebra topic. These model working out are displayed when the user submits an incorrect answer. \\

The could requirement has not been completed. This is because requirements 8 and 9 have not been completed so work for requirement 10 has not been started. \\

As non-functional requirements are difficult to test, it is difficult to determine in absolutes whether these requirements have been met. Any tests conducted with 12 year olds to determine their comprehension of the questions in the application would not have been representative of the population as a whole, this method was not used to attempt to assess the completion of the non-functional requirements. Instead, user testing was conducted with colleagues who work with students of GCSE age range. Their feedback was incorporated into the development process as explained in the testing section of this report. Their feedback on the final application was that the functional requirements have been completed for almost all students who are in the English year 11, and the majority of students in the English year 9. Because of this, these functional requirements have been deemed completed for this project. \\

\subsection{Results Overview}

By the definition of success laid out in the project specification, as well as discussed in the introduction to this report, this project was successful. All of the must requirements were completed, as well as one of the three should requirements, although the could requirement was incomplete. \\

Overall, the project produced a proof of concept that the MyMaths homework functionality could be turned into an application, as was the end goal. This project could be released on the Google play store as it is, although if this project were to continue the application would not be uploaded to the Google play store for another two to three months of full time work. This extra time would be spent completing the should requirements, and working on the could requirement for content which is examined on the foundational exams. \\

This project could have been improved with better project management throughout the first term of development. The project fell behind early in development due to personal issues on behalf of the author. Alongside said personal issues, time management was poor throughout the early stages of this project, and project management techniques which were key in improving the project in term two were not used. Improvement was shown in project management the further into the project development got. Improvements were made to the authors time management skills, such as dedicating specific time each week to work on the project. These improvements had a large impact on the final product, and counteracted the previous loss of development time from the first term. \\

Advanced software engineering techniques have been included in this project, for example the use of a factory class for generating test questions. These were included where appropriate. \\

This project was worthwhile as it highlights that more work can be completed on mobile GCSE revision applications, whilst also displaying the amount of work required to build an app of this kind. Students would benefit from an application, endorsed by teachers, which aided in a student's education in GCSE mathematics, whether that application is similar to one built by this project, or another application with slightly different goals like Khan Academy. Reducing the barrier of entry for students to be able to improve their education only requiring access to a smartphone could increase GCSE grades, as most students have regular access to a smartphone, as referenced in the introduction of this report. \\

\section{Conclusion}
\label{section:conclusion}

\section{Future Work}
\label{section:futureWork}

This project was intended as a proof of concept, trying to create an android application which mimics the homework feature from MyMaths. The final product from this project is not perfect, and could be extended and released to the Google play store. This section of the report will outline key changes which would be made to the project if restarted. After, it will explain extensions which would be made to this project if it were to be continued. \\

\subsection{Project Improvements}

These improvements are the changes that would be made to the application if the project was started again from scratch. 

\subsubsection{Question Storage}

In the current implementation of the application questions are hard coded into the application. This has several disadvantages. One of the disadvantages of hard coding the questions into the application is that any changes made to questions would only be available to users through updates to the application. This delays question changes getting to users and would cause lots of small app updates to fix small problems in questions. Another disadvantage to this hard coding question storage is that users cannot add their own questions to the pool of questions in the application. This is because users do not have access to, and cannot change, the source code of the application. \\

There are two methods which could be implemented to store questions in the application, both with benefits and drawbacks. Both methods would require a slight change in the logic of how questions are read and displayed. \\

The first method of question storage would be to write the questions to a JSON file which is stored locally on the system. This would allow the manipulation of questions (adding questions, changing current questions, etc) locally. This would enable users to add questions just by appending an element to the JSON file, which could then be implemented easily. It would only require a small change of the current question displaying logic. This is because the current question display logic is written to take an Array list storing string types and iterate through it until all elements are displayed, this could be converted to iterating through the data in a JSON object until there is no more data. Also, the data would not be rewritten each update. A second JSON file can be added which contains just the user added questions, and these questions can be appended to the questions JSON file with each update. \\

The second method of question storage would be to insert elements into a database. This has the same benefit of being local storage and so adding user questions is as easy as inserting the question information into the database. SQLite is built into Android and so it would be easy to implement. It would, however, require a complete rewriting of the question displaying logic. This is due to the question logic iterating through the Array lists. Instead, the question displaying logic would query the database, fetch some relevant data, and find a way of displaying that data and generate the random numbers necessary as per the requirements of this project. \\

If this project was restarted then the method which would be implemented would depend on the amount of question data. For smaller question amounts the JSON file would be used. This is because iterating through a JSON file until the right question is found would not add too much time onto question generation. The convenience of creating a working implementation of the JSON storage method would be too great to use the database method. Although, when the JSON file gets large, and lots of questions are contained in it, iterating through questions until the correct question is found would be time consuming. In this situation the database method would be used. SQLite would be much faster at locating a specific row in the database than iterating through the JSON file. \\

\subsubsection{Accessibility Improvements}

If this project were to be restarted then accessibility issues would be much higher on the priorities list than it was. Questions could be played in audio form for those users who are blind. Audio files could be saved to the users device and concatenated to display any questions that could be generated from the application. Also, when an answer is inputted that number could be spoken aloud, or in the case of a multiple choice question the selected answer could be read aloud. This would enable a user which has visual impairment to be able to use the application. \\

There could also be a setting which increases the font size of all text in the application, and would also increase image size. This would also aid users with visual impairments, and would be built into the application from the first version. \\

\subsubsection{Content Grouping}

Content could be grouped into appropriate ages as well as modules. This would aid younger students, those ranging from academic years 7 to 9, to utilise the application to revise. This would increase the target audience of the application, and aid more students in their understanding of GCSE Mathematics. \\

Currently, any students in that age range who wish to use the application could get questions above their skill level. This is counterproductive to the aims of this application, as students would be frustrated that they cannot answer a given question as they have not covered the content, and they are less likely to continue to use the application. Instead, with content grouped into appropriate academic years, as well as modules, students can be certain that questions that they are tackling in the application are appropriate to their skill level. \\

\subsection{Project Extensions}

Project extensions are features which would be added to the application if this project continues. 

\subsubsection{Currently Uncompleted Requirements}

If this project continues the first thing that will be completed are the three uncompleted requirements. These are requirement 8, 9, and 10. This will mean that the application will track users recent feedback, display model answers when the user gets an answer incorrect, and there will be some kind of classroom like learning resource. \\

These requirements were in the original scope of the project but due to the project running out of time they were incomplete. This is why they would be completed first if extra time was added onto this project, as requirements 8 and 9 are already in the prototype phase. 

\subsubsection{User Validation}

User validation would be a key feature which would be added to the application. Two types of user accounts would be created, those available for students and those available for teachers. Student accounts will be able to attempt any questions from any module in any year group. They would have full access to all testing and learning content at any time. Teachers will have the ability to create classes which will contain a group of students from the same academic year. \\

User validation was outside of the scope of the current project as there are many legal issues with storing user data. This application would have had to register under the data protection act and ensure that all data collection is compliant under the new GDPR laws as this application would be aimed at being used primarily in the United Kingdom. It was deemed that the potential legal issues which come with storing user data was not worth the risk for a project of this size and scope. This is why if the scope was increased and time was added on then these legal issues will be tackled. \\

A user would need to be able to register an account with an email, whether that is a school specific email or not, and use the application. A single sign on (SSO) feature can also be added to allow users to sign in with other popular accounts such as facebook and google accounts. This would shift the burden of sign on security to the validating third party, although would not be practical for all cases. A user may not have a third party account and so there would need to be the ability for a user to register an account with their email as well. \\

\subsubsection{User Added Questions}

A feature which would be useful for users of this application is users being able to add their own questions to the question set in this application. This was deemed outside of the scope for the current project as this is not a feature available by any of the major GCSE Mathematics revision applications. This would be added after the question storage technique would be changed, as currently there is no way to automate the question adding process for a student due to how questions are stored. There is also no way to have user specific questions as the questions are built into the source code for the application. \\

Once the question storage method is changed from the current method used to one of the two other methods discussed above, giving users the ability to add questions would be implemented. This feature would be added only for user initiated testing and would be analysed by the statical tools separately to the questions added into the application by the author, or a student's teacher. This is to prevent students from abusing the question adding process to add questions which have a single, simple answer, or are much simpler than other questions in the same skill level. If the questions were not analysed separately then a student could abuse the question adding mechanism so that they get 100\% in a homework or revision set. \\

This would be implemented by adding an activity which would allow the user to input details for a new question. Users would utilise radio buttons to select the module and submodule for the specific question, as well as the year if that had been implemented when user added questions is. There would be a text box where users could type out the question text, and a specific set of characters would be used to represent variables which the application has to generate. This set of characters would then be searched for using regular expressions. \\

\subsubsection{MyMaths Homework Feature}

Given more time with this project a homework feature, similar to that of MyMaths, could be implemented. This would allow teachers to set classes revision tests to be completed by a given date, and with over a specific percentage score. The students would then be notified of the homework with a push notification, as well as something visual within the application, preferably on the homepage of the application so students do not miss it. \\

Students could then attempt the specific homework with their percentage score being sent back to the teacher to analyse. If a student achieves a score below the passing grade for that homework, then they could reattempt the homework, or look at some revision materials for that specific topic. These homework questions would be separate from the user added questions, for reasons of dishonesty mentioned above. \\

Teachers could also be provided with the ability to select specific questions for a topic, and change the appearance probability. This would grant teachers the flexibility to customise a homework specifically for their class. \\

\section{Legal, Social, Ethical, and Environmental Issues}
\label{section:issues}

This project did not face many legal, social, ethical, or environmental issues as it was a safe topic. There were no environmental or social issues faced for this project. There were limited legal and ethical issues which will be discussed under their respective subheadings. 

\subsection{Legal Issues}

There was one legal issue faced throughout this project. This issue was related to the tools which were used. As the final source code was not all written by the development team for this project; some of the code was generated through the tools used to aid this project, for example any code pre-generated through the Android Studio IDE was included in the final source code. This could create legal issues due to copyright laws. \\

This is why the licensing for all the tools used is linked below as a reference, and relevant sections are quoted to ensure that there are no legal issues with this project. \\

\subsubsection{Android Studio}

Android Studio is built on top of the IntelliJ platform. It shares the same license as IntelliJ, as is shown in this comment on a blog post by the creator of IntelliJ Dmitry Jemerov (Yole in the blog post) [17]. The IntelliJ IDEA (the IntelliJ IDE) is licensed under two licenses, but as is explained in the blog post, the licensed version of IntelliJ IDEA that is used as the basis of Android Studio has the Apache 2 license. The relevant sections of the Apache 2 license are as follows: "You may reproduce and distribute copies of the Work or Derivative Works thereof in any medium, with or without modifications, and in Source or Object form"[19]. This allows for the use of the IntelliJ and Android Studio IDEs for commercial and independent development which this project falls under, and so there are no legal issues with using Android Studio as the IDE for this project. \\

% Find the licensing for Android Studio, the link which proves that it copies the IntelliJ license, and then the 
% relevant parts of that license. 

\subsubsection{Git}

The majority of git is released under the GNU public license [2] which enables copying, modifying, or distributing git under any circumstances. The relevant sections of the GNU licence (from the preamble): "By contrast, the GNU General Public License is intended to guarantee your freedom to share and change free software--to make sure the software is free for all its users.". This enables free use of git for this project. \\

Small sections of git are not licensed under the GNU public license, but under the GPL (GNU Lesser Public License) [3]. Relevant sections from the GPL preamble: "By contrast, the GNU General Public Licenses are intended to guarantee your freedom to share and change free software--to make sure the software is free for all its users.". This enables the free use of git for this project. \\

As these are the only two external tools used throughout this project, and the only tools which could have impact on the source code of this project, then there are no legal issues with this project. Any other materials were produced by the author specifically for this project, and so there are no legal problems with using these materials. \\

\subsection{Ethical Issues}

This project only contains one key ethical issue, which is to do with the testing of the final product. Testing was conducted on colleagues who gave verbal and written consent to test the application. User testing was conducted where the user was observed using the application. This testing method makes this project one which does not require ethical review under department guidelines which can be found here [1]. "Student projects with primarily an educational purpose, for example asking peers or family members to test software as part of your course. Such projects should not include any form of deception or coercion or involve vulnerable groups (e.g. schoolchildren)." As testing was not conducted on school children for this project, instead user testing was conducted on colleagues over the age of 18, this project is ethically sound. 

\section{References}
\label{section:references}

[1] - https://warwick.ac.uk/fac/sci/dcs/teaching/ethics (Accessed 22\textsuperscript{th} March 2020) \\ 

[2] - https://github.com/git/git/blob/master/COPYING (Accessed 22\textsuperscript{th} March 2020) \\ 

[3] - https://github.com/git/git/blob/master/LGPL-2.1 (Accessed 22\textsuperscript{th} March 2020) \\  2020)

[4] - https://www.mymaths.co.uk/ (Accessed 22\textsuperscript{th} March 2020) \\ 

[5] - https://www.mymaths.co.uk/help.html (Accessed 22\textsuperscript{th} March 2020) \\

[6] - https://www.khanacademy.org/ (Accessed 22\textsuperscript{th} March 2020) \\

[7] - https://www.khanacademy.org/math/trigonometry/trig-with-general-triangles/law-of-sines/v/law-of-sines (Accessed 22\textsuperscript{th} March 2020) \\

[8] - https://apps.apple.com/gb/app/khan-academy/id469863705 (Accessed 22\textsuperscript{th} March 2020) \\ 

[9] - https://play.google.com/store/apps/details?id=org.khanacademy.android (Accessed 22\textsuperscript{th} March 2020) \\

[10] - https://qualifications.pearson.com/content/dam/pdf/GCSE/mathematics/2015/specification-and-sample-assesment/gcse-maths-2015-specification.pdf (Accessed: 6 October 2019) \\

[11] - https://mathswatch.co.uk/ (Accessed 25\textsuperscript{th} March 2020) \\

[12] - http://www.mathswatch.co.uk/subscription/4594745491 (Accessed 25\textsuperscript{th} March 2020) \\

[13] - https://developer.android.com/studio/run/emulator (Accessed 26\textsuperscript{th} March 2020) \\

[14] - https://github.com/AlphaMustang/thirdYearProject (Accessed 28\textsuperscript{th} March 2020) \\

[15] - https://developer.android.com/guide/components/activities/intro-activities (Accessed 28\textsuperscript{th} March 2020) \\

[16] - https://developer.android.com/guide/components/fragments (Accessed 28\textsuperscript{th} March 2020) \\

[17] - https://blog.jetbrains.com/idea/2013/05/intellij-idea-and-android-studio-faq/\#comment-4939 (Accessed 29\textsuperscript{th} March 2020) \\

[18] - https://www.jetbrains.com/idea/features/ (Accessed 29\textsuperscript{th} March 2020) \\ 

[19] - https://www.apache.org/licenses/LICENSE-2.0 (Accessed 29\textsuperscript{th} March 2020) \\

[20] - https://www.gov.uk/government/speeches/reformed-gcses-in-english-and-mathematics (Accessed 29\textsuperscript{th} March 2020) \\

[21] - https://qualifications.pearson.com/en/qualifications/edexcel-gcses/mathematics-2015.html (Accessed 29\textsuperscript{th} March 2020) \\

[22] - https://www.pewresearch.org/internet/2018/05/31/teens-social-media-technology-2018/ (Accessed 30\textsuperscript{th} March 2020) \\

[23] - https://www.comscore.com/Insights/Presentations-and-Whitepapers/2017/2017-US-Cross-Platform-Future-in-Focus (Accessed 30\textsuperscript{th} March 2020) \\

[24] - https://gs.statcounter.com/os-market-share/mobile/worldwide (Accessed 30\textsuperscript{th} March 2020) \\

[25] - http://coolt.ch/notizen/wp-content/uploads/2016/02/Head-First-Android-Development-2015.pdf (Accessed 31\textsuperscript{th} March 2020) \\

[26] - https://developer.android.com/docs (Accessed 31\textsuperscript{th} March 2020) \\

[27] - https://github.com/AlphaMustang (Accessed 31\textsuperscript{th} March 2020)

\end{document}