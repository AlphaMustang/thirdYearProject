\documentclass{article}

\usepackage{subcaption}
\usepackage{float}

\begin{document}

\section{Abstract}

\section{Introduction}

\subsection{Project Motivation}

\subsection{Problem Outline}

\subsection{Project Specification}

\subsection{Report Structure}

\section{Background}

\section{Methodology}

Go into detail about the scrum methodology.

\section{Design}

Discuss the control flow of the program, why it works the way it does etc...

\section{Main Body}

\section{Key Decisions}

\section{Project Management}

\subsection{Schedule}

A schedule was created early, and updated throughout the project to represent and handle the amount of work which had been completed. Two key versions of the schedule were displayed to various stakeholders throughout the project in key documents. These key documents were the project specification and the progress report. 

\subsubsection{Project Specification Schedule}

At the beginning of this project a schedule was created. This schedule aimed to complete all of the requirements created in the project specification[x]. The schedule was created using a technique common in industry which involved some numbered cards. Each requirement was reviewed. Each review consisted of analysing the relative difficulty of the requirement with respect to all of the other requirements. The difficulty was quantified using the numbered cards, the cards would follow a logarithmic pattern, doubling with difficulty each time. The lowest card was 1/4 and the largest was 16. Any requirement larger than 16 was then broken down into sub-requirements, these sub-requirements would then in turn be broken down if they were above 16 difficulty. \\

This relative difficulty in requirements was used to create a rough schedule for the project, assuming all requirements would be completed. This schedule was the one included in the original project specification. This schedule can be seen as a Gantt chart in figure[x].

\subsubsection{Progress Report Schedule}

The progress report submitted on 24\textsuperscript{th} November 2019 updated this schedule, once it was realised that development was slower than initially intended through term one of this academic year (October 2019 - December 2019) due to personal reasons. The new schedule, referred to from now on as the Christmas schedule, again made the assumption that every requirement would be completed before the deadline. To mitigate against the slow development conducted through term one, the Christmas schedule increased the workload of the project throughout the Christmas period (8\textsuperscript{th} December 2019 - 8\textsuperscript{th} January 2020) and throughout term two of the project. 

\subsubsection{Schedule Evaluation}

The image displayed below as figure [x] is an assessment of the completion of the final requirements in respect to the project specification schedule. 


\subsection{Tools}

This project utilized several tools to aid in the development of this Android application. Two of the main tools used to add value to this project are outlined below, one from a development point of view and one from a project management standpoint. 

\subsubsection{Android Studio}

Android Studio was used as an IDE (Integrated Development Environment) throughout development as it comes with many useful built in features. These features are specifically designed with android app development in mind. Examples of these features are: 

\begin{itemize}
	\item Rendering engine which converts XML into how it will appear in the application itself
	\item Android phone emulator
\end{itemize}

The rendering engine enables both the ability to edit the XML tags and see the effects in real time, as well as the ability to edit directly into the rendering engine with a drag and drop feature. This enables quick prototyping for new pages through the use of the drag and drop feature whilst also allowing a much larger degree of control through the ability to directly edit the XML.\\

Use of the android phone emulator enabled quick and efficient testing cycles. When a feature was initially developed, it could be tested on a wide range of devices through the phone emulator. This add on enabled the emulation of any modern phone through the ability to download files specific to that device. The app could then be tested on an accurate test bed, seeing how the app would look and work on a mobile device. Tablets could also be emulated, which helped with ensuring the application worked on a wide variety of devices.\\

The Android Studio IDE greatly increased the efficiency of development. This was encouraged through the notion that it was only designed specifically for android development, and so could focus entirely on that. Being built upon the IntelliJ platform meant that code editing was incredibly smooth and intuitive. \\

\subsubsection{Git}

Git was used as version control for this project. This enabled access to the project from any device, anywhere, anytime. It also enabled the ability to revert back to an earlier version of the project if something went wrong. This meant that development could take place without fear of failure, as if anything went wrong the project could just be reverted to the last working state.\\

The repository for this project can be found at[x]. This repository is public and can be accessed by anyone. This enables anyone to see the code that would be running on their machines.\\

Git also logs the time when a change is made to the code. This enabled precise tracking of requirement completion and aided project management. Git commit messages for this project were all deliberately helpful. Git commit messages were written so that they would explain the changes made to the codebase, so they could be logs of when project requirements were completed. The git commit messages for this project can be viewed and give a detailed timeline of when requirements were completed and features were added. This was incredibly useful as it gave updates of any changes made to the codebase as soon as it was available to anyone who follows the repository, and also was used to create a timeline of when requirements were completed for this final report. \\

% Move into the methodology section. Certainly the information about the notes file.
\subsection{Weekly Work????}

Weekly meetings were held with the project supervisor to keep key stakeholders in the project up to date with development. Alongside weekly meetings, access was given to the project supervisor to the  github repository early in development. This enabled the project supervisor to directly monitor completed work. \\

A document in the github repository called notes.txt is a file which was edited after every development session was completed. This file contained notes on what was intended to be completed during each development session, as well as what was actually completed. A substantial amount of time was spent at the beginning of early development sessions on remembering what had been completed throughout the previous development session, what state the code was in, and what needed to be completed in this development session. This file cut down the time spent at the beginning of development sessions looking through git previous git commits, and so overall this file caused development to be more efficient. \\


\section{Results}

\section{Conclusion}

\section{Future Work}

\section{Legal, Social, Ethical, and Environmental Issues}

\subsection{Legal Issues}

\subsection{Social Issues}

\subsection{Ethical Issues}

\subsection{Environmental Issues}

\section{References}

\end{document}