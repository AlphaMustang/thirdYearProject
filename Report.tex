\documentclass{article}

\usepackage{graphicx,color,tabularx,amsmath,amssymb,fancyvrb,soul}
\usepackage{subcaption}
\usepackage{float}

\setlength{\parskip}{1em}

\title{CS310 Computer Science Project \\ Final Report \\ The Creation of an Android Application to Aid a Student's Revision of Edexcel GCSE Mathematics}
\author{Ryan O'Hara \\ u1702761}

\begin{document}

\maketitle

\newpage

\tableofcontents

\newpage

\section{Abstract}
\label{section:abstract}

%
%
%
%
%
%
%

\section{Introduction}
\label{section:introduction}

The introduction of this report aims to complete three tasks. The first task is to provide an understanding of the problem which this project wishes to solve. The second task is to explain the features required by this project to be deemed successful. These features were laid out in the project specification document[appendix 1] which was produced at the beginning of the project. Finally, this project aims to provide the reader with an understanding of the report structure, giving brief descriptions of each section of the report to explain their purpose.

\subsection{Problem Outline}

The difficulty of GCSE Mathematics recently increased due to a change in specification by Edexcel in 2015. This change aimed to include content in the GCSE Mathematics exam which was previously a part of AS level Mathematics. This increase in difficulty of GCSE Mathematics was due to the UK Government's reform of the English and Mathematics GCSEs in 2013[1]. The relevant section from the Department of Education's statement is as such: "The new mathematics GCSE will demand deeper and broader mathematical understanding ...  The new mathematics GCSE will be more demanding". This demand for deeper and broader mathematical understanding enforced a syllabus change of GCSE Mathematics. This syllabus change was produced by Edexcel and started teaching in 2015, and examining in 2017[2]. \par

This increase in difficulty for GCSE mathematics is achieved through the introduction of more content. This content is initially covered in the classroom, but due to more content needing to be taught, less time is spent revisiting and revising content at the end of the course. This leaves students in the position of having to revise topics for the final exam at home, typically without the support of a teacher or tutor. \par

95\% of American teenagers have access to a smartphone[3] a study from 2018 stated. Also, a recent study conducted in 2017 stated that "Recent statistics show that around 69\% of the total digital media time spent by Americans was taken up by mobile devices and 87\% of the total mobile time was spent in apps in December 2016"[4]. More students than ever before have access to smartphones, and they are spending the majority of their time on these smartphones in apps. More students have access to a smartphone than have access to a desktop or laptop computer as of 2017[3]. \par

Because of this increase in the availability of smartphones, students are now turning towards technology to aid with their learning and revision for GCSE exams. This project aims to build a mobile application which can be used on the Android mobile operating system, which is the majority operating system (OS) for smartphone devices with a market share of 73\%[5]. This allows the application to be available to the majority of students studying for GCSE mathematics. \par

This project aims to create an application which focusses on testing students current knowledge instead of teaching them new information. This will help students to revise for their GCSE Mathematics exam. It could be used alongside classroom teaching, or other less conventional teaching methods which are used to replace the classroom environment, such as Khan Academy. This application aims to focus on testing skills already gained by the user from a classroom like environment, which will prepare the user for sitting their Edexcel GCSE Mathematics exams. \par

This application will have to provide users with the ability to practice skills gained from classroom learning. It must contain questions of GCSE exam standard so that students can effectively prepare for their exams using the application. It must also have the ability for users to input answers, and it must mark the answers and provide feedback to the users. A set of requirements which were created to complete these tasks are available below in the project specification subsection of this introduction. \par

\subsection{Project Specification}

This project aims to create an android application similar to the MyMaths homework functionality. This will enable student's to load exam questions in a specific topic in Edexcel GCSE Mathematics and input an answer. This answer will then be compared to the original answer, and the student will get a visual response as to whether they answered the question correctly or incorrectly. Where the student answered the question incorrectly, the correct answer should be displayed to the student so they can compare their answer with the correct one. \par

To achieve this task a set of requirements were laid out during the project specification[x]. These requirements follow the MoSCoW system for writing requirements. The MoSCoW system groups requirements into four classes, must, should, could, and won't. All of the must requirements are vital to the success of the project, they need to be completed by the end of the project. At the completion of all of the must requirements, a minimum viable product (MVP) should be created. Should requirements are features which are not essential to the completion of a MVP but are features which are expected in the final product. Could requirements are features which are not expected in the final product, but if all other requirements are completed before the deadline, the could requirements will be implemented. Won't requirements are features which will not, under any circumstances, be present in the final product. \par

The requirements present in the project specification are as follows: \par

The Android application which will be created for this project must: 

\begin{enumerate}
	\item Have a start page with several clickable buttons
	\item Be able to run on Android devices with Android versions between Android 6 and Android 10
	\item Have a list of each module found in Edexcel's GCSE Mathematics Specification[6], with each module being selectable
	\item Generate revision questions where numerical values are generated by Java's inbuilt random number generator and make sense
	\item The user has the ability to input numerical answers to the revision questions generated (requires 4)
	\item When answers are submitted the user must receive feedback relating to whether	their answer matches the correct answer for the question, and where the user’s answer is incorrect, display the correct answer. \\
	
	The Android application which will be created for this project should:
	\item Have a variety of testing difficulties, ranging from easy to hard for each module
	\item Have a way of tracking user progress over a number of days or tests per module (requires 5)
	\item Display model methods for working out the correct answer for a question where the user inputted an incorrect answer (requires 6) \\
	
	The Android application which will be created for this project could:
	\item Include resources similar to those which would be used in a classroom environment to teach the user about a topic
\end{enumerate}

Project success was also defined in the project specification. To be deemed successful, the final product produced by this project will have completed: 

\begin{itemize}
	\item all of the must requirements defined above
	\item at least one of the should requirements defined above
	\item any value of could requirements defined above
\end{itemize}

The target audience for the final product of this project is students sitting Edexcel GCSE Mathematics examinations. The average age of students sitting this exam is 16 years old. Although, students as early as year 7 (age 11 - 12) begin learning content which is examinable in the GCSE Mathematics exam. The app will be primarily designed for 15-16 year olds, but needs to be accessible to students as young as 12. This means there are a group of non-functional requirements which were not outlined in the project specification but were followed during development which are shown below: 

\begin{itemize}
	\item The language used in the application should be understandable to 12 year old school children
	\item The application should be simple and minimalistic as to not confuse new users
	\item A 12 year old who has not previously used the application but is familiar with smartphones should be able to use the application effectively with minimal prompting
\end{itemize}

These functional requirements enable anyone learning the content for the Edexcel GCSE Mathematics exams to be able to use the application and benefit from it.

\subsection{Report Structure}

This report will start by running through the background to this project. The background section will outline the current landscape of the problem, what has currently been achieved by other people. It will be followed by an explanation of the software development methodology which was used throughout the development of this application. This methodology section will explain how the work was completed from a project management point of view, for example it will outline a typical development session. \par

Section \ref{section:design} outlines the design of the application which will provide a high level overview of the application. This high level view will explain the control and data flows of the program, and any specific design choices which was made throughout the development of this project. Section \ref{section:implementation} discusses the specific implementation details of the project. For example, how the questions were created in the application. Testing will be discussed in section \ref{section:testing}. This will consist of outlining how testing was conducted, and the results of user testing. \par

Risks encountered throughout the project will be evaluated in the risks sections, which is section \ref{section:risks}. An outline of the risks will be produced, followed by how the risk was handled, which will then be followed by the outcome of the risk. Then project management will be discussed in section \ref{section:projectManagement}. This section will be a discussion on several details of the project that cannot be seen from the source code, for example the development schedule, or weekly meetings with key stakeholders. These details do not appear in the source code but impact the project greatly. The results section follows the project management section, where the requirements will be evaluated and the success of the project will be determined. \par

The conclusion will be section \ref{section:conclusion} which will summarise key findings in the report. This will be followed by the future work which could improve this project, and turn it from a proof of concept into an application which would be able to be released on the Google play store. Section \ref{section:issues} will discuss the legal, social, ethical, and environmental issues associated with this project. \par

The final section in this report will be the references section, with a section number of \ref{section:references}. Harvard referencing style is adhered to throughout this report.\par

\section{Background}
\label{section:background}

There are two main GCSE Mathematics revision tools in use today that are used in schools. Both of the main revision tools are explained in depth in their respective sections. There are also a number of other independent revision tools which are not actively used in schools, but are able to be downloaded and used on mobile devices. Maths Watch is an older tool which students used to use to learn content and revise for their exams. There have been small attempts to modernise Maths Watch, to limited success. Finally, this section will discuss where this application fits into the maths revision ecosystem. \par

\subsection{MyMaths}

The first tool is MyMaths[7] which is a website which aims to enrich a student's classroom learning experience. MyMaths main use case is to supplement the material covered in class by a teacher. The main feature of MyMaths is the online homework feature. This feature enables teachers to set homework to their class. This homework covers a specific topic, for example Venn diagrams, and the students can answer questions on this topic. Example homework questions on the topic of Venn diagrams can be viewed as figure \ref{figure:mymathsHomeworkQuestion1}.
\begin{figure}[H]
	\centering
	\includegraphics[width=0.9\linewidth]{./data/mymathsQuestion1.jpg}
	\caption{Venn diagram homework questions.}
	\label{figure:mymathsHomeworkQuestion1}
\end{figure}
The student's scores are then generated (as a percentage) and sent back to the teacher. \par

The values selected in the homework are unique each time a student attempts that specific homework. This prevents a student from memorizing a specific answer to pass the homework. A teacher can also see the number of attempts that a student has currently taken for the homework, and their highest score. Giving a teacher the ability to see the number of attempts that a student has at questions on a particular topic highlights where a student is struggling and can enable one on one teaching to boost the student's knowledge in that particular topic. A student can also attempt a homework which has not been set by their teacher. This allows a student to revise any topic in GCSE Mathematics on their own without a teacher. \par

MyMaths also gives a student the ability to take a lesson in any topic in GCSE Mathematics. This lesson walks through an example question in the topic, and then provides practice questions which are of a similar style to the homework questions. These questions are then marked, and where the student answers the question incorrectly, the correct answer is displayed next to the answer box. \par

MyMaths has an iOS and Android application for tablets, but not a specific application for smartphones[8]. This prevents students who own their own smartphones but do not have access to a tablet on a regular basis from revising using MyMaths. \par

\subsection{Khan Academy}

The second main GCSE Mathematics revision tool is Khan Academy[6] which is a non-profit organization which aims to "provide a free, world-class education for anyone, anywhere"[9]. Khan Academy's main function is to teach new information to people. It does this through the use of short videos covering particular concepts. For example, there is a six minute video covering how to find a side of a triangle using the sine rule[10]. These videos aim to replace classroom teaching for those who are unable to attend school. There is a video for every topic in GCSE Mathematics and so a student could exclusively use Khan Academy to learn the content for their exams. Khan Academy also has a testing feature, where for a certain topic four questions come up which the user has to answer. This is, however, not available for all topics, and the user is limited to answering four questions at a time. \par

Khan Academy has applications in both major app stores, the Apple app store[11] and the Google play store[12]. This app runs on all major devices and has the same functionality as the website version of Khan Academy. Users can create accounts for free which keeps track of their progress and can access videos for all topics available on the website for free. \par

\subsection{Independent Revision Tools}

There are several other mathematics revision apps on both app stores, examples of which can be seen below as figure \ref{figure:appStoreApps}. These apps mostly focus on testing using some form of revision questions. These questions are either limited to multiple choice questions, or questions which have the same numbers each time they appear. This leads to two problems: random guessing for the multiple choice questions, or answer memorization for the questions which do not change their numbers. This leads to these applications not being endorsed in schools. \par

\begin{figure}[H]
	\centering
	\begin{subfigure}{0.8\textwidth}
		\frame{\includegraphics[width=0.9\linewidth]{./data/appleAppStoreApps.png}}
		\caption{Revision applications on Apple's app store}
	\end{subfigure}
	\hfill
	\begin{subfigure}{0.8\textwidth}
		\frame{\includegraphics[width=0.9\linewidth]{./data/googlePlayStoreApps.png}}
		\caption{Revision applications on the Google Play store}
	\end{subfigure}
	\caption{Revision applications on both the Apple app store and the Google play store}
	\label{figure:appStoreApps}
\end{figure}

\subsection{Maths Watch}

Maths Watch[13] is an older tool which was used for teaching purposes in schools. This is a service which provides educational videos for students to learn from. It is designed to replace both classroom teaching and as a homework tool. Mathswatch is a subscription service, which only schools can subscribe to. If a student's school is not subscribed to Maths Watch then a student cannot get access to it. \par

Maths Watch has two main methods for distributing their video content. The first method is through DVDs which are posted to schools which buy them, and the students can then get these discs from the schools with all of the videos on them. The second method is through their online portal which provides the ability for paid users to stream video lessons. The Maths Watch service also states on the subscription page of their website[14] that they provide "A bank of 1000s of interactive exam-style questions for students' independent work with instant feedback". This is similar to MyMaths where a user can submit their answers and get feedback on whether they were correct, and if they were incorrect have the correct answer displayed. \par

The Maths Watch service is used by students all throughout the UK and is an incredibly valuable tool to aid their learning. Unfortunately, Maths Watch does not provide a mobile application on either of the major app stores (Apple iOS app store, or the Google Play store). As well as this, only schools can purchase Maths Watch, so if a students' school is not a Maths Watch partner, then the student cannot get access to Maths Watch. \par

\subsection{App Development}

Research into Android application development consisted of: 

\begin{itemize}
	\item Working through Head First Android Development[15]
	\item Looking through the Android Documentation [16]
	\item Making small Android apps available on github [17]
\end{itemize}

These tools were utilised in learning the basics of Android application development before any code was written for this project. The Android documentation helped with any programming issues within the code during the development of this application. 

\section{Methodology}
\label{section:methodology}

This section will describe how the work necessary for this project was completed. It will describe the process with which the final application was developed at a high level, and section \ref{section:implementation} will explain the lower level ideas with examples from the final application's source code. \par

The process with which work was completed within this project, often referred to as the methodology, changed throughout this project. The project specification stated that "This project will make use of the scrum software engineering methodology.". The scrum methodology was not followed throughout the early stages of this project. Instead, a non-specific agile methodology was utilised, as each development cycle planned only the next requirement to be added as a feature. Each development cycle did not create user stories, or have a pre-defined length of between one and four weeks, which are both key features of the scrum methodology. \par

The non-specific agile methodology utilised for the early stages of this project consisted of selecting the next requirement to implement into the program. Any decisions that needed to be made about the implementation would be made at this stage and a small plan would also be created detailing how to implement the requirement. Finally, the code would be written for the requirement to the plan. Once the requirement was completed, there would be no review, development would move onto the next requirement. \par

Monitoring and controlling was continuously conducted throughout this project. Midway through the project it was noticed that the methodology was not being followed as was set out in the project specification. This problem is discussed later in the risks section of this report which is section number \ref{section:risks}. This deviation from the initial project plan was disruptive to the progress of the project, and so the methodology used for this project was changed midway through the project. It was still a non-agile methodology as no user stories were written, but each development (sprint) cycle consisted of one week, with specific time being dedicated each week to development. The other major change to the methodology was including sprint reviews. These two changes meant that the methodology could be viewed as a variation on scrum which used requirements instead of user stories.\par

The final methodology change for this project was implemented during the last month of the project to aid with writing this report and continuing development in parallel. For this section of development focussing on both the report and continuing to implement requirements was difficult. Initially, each section of the report were aimed to be written in parallel to decrease the amount of idle time in the writing process due to bottlenecks. For example, sections such as design required figures to be created which could not be created in a writing environment. This meant that if multiple sections were not worked on in parallel then there would be idle time. To fix this multiple sections were worked on simultaneously when writing the report. This caused its own issues of an increase in time spent context switching, which can be viewed as wasted time. Once the context switching increased above a specific boundary it was decided to switch the report writing from following the scrum methodology to Kanban. \par

This change in methodology manifested itself in two key ways. The first was using Trello[18] as a way of visualising workflow. A Kanban board was set up, and can be seen as figure \ref{figure:trelloBoardOG}. 

\begin{figure}[H]
	\centering
	\frame{\includegraphics[width=0.9\linewidth]{./data/trelloCopy.jpg}}
	\caption{Trello board used as a to-do list for this report.}
	\label{figure:trelloBoardOG}
\end{figure}

This Trello board was made to effectively visualise the work that had to be completed for this report to be deemed completed. Work is pulled from the 'Things To Do' section into the 'Doing' section. Once the work is completed, it is moved into the 'Done' section. The second change to implement Kanban was the introduction of work in progress limits. Work in progress limits reduce the amount of time spent context switching by limiting the amount of tasks which can be worked on at any one point in time. There is one work in progress limit set in figure \ref{figure:trelloBoardOG} and is denoted by a black line under the 'References' card in the 'Doing' section. This work in progress limit decreased the amount of time spent context switching, and also narrowed the focus of each development session. Instead of having the choice between working on several sections of the report, the choice was between two sections of the report at any one time. This narrowed focus increased the productivity of the report writing, and also increased the quality of the text written. 


\section{Design}
\label{section:design}

This application had a target audience of school children between the ages of 12-16 years old. This meant that any design would have to be simple and intuitive to prevent users from getting 'lost' within the application. The graphs for both the control flow and the data flow for this application are acyclic as is shown by the two figures below, figure \ref{figure:applicationControlFlow} and figure \ref{figure:applicationDataFlow} respectively. Users can only follow certain paths, and will only end up in the same place by going backwards using the inbuilt backwards button in Android, so they cannot get confused about the layout of the application. Also, the number of pages was kept to a minimum for the application to function for the same reason. This minimalist design aims to prevent confusion by keeping the application simple to understand and use. \par

\subsection{Control Flow}

Figure \ref{figure:applicationControlFlow} is a graph for the control flow of the application. Each edge is bidirectional, the user can move forward by interacting with the application, and backwards by hitting the back button built into the Android operating system. The user interacts with the application in one of three ways, either by pressing a button on their screen, selecting an element from a list in front of them, or entering text into a text field. \par

\begin{figure}[H]
	\centering
	\frame{\includegraphics[width=0.6\linewidth]{./data/applicationControlFlow.png}}
	\caption{The control flow of the application.}
	\label{figure:applicationControlFlow}
\end{figure}

The only user choice in direction is at the start page where a user could select between the information page for the application and the modules list. Each node in this graph represents an Android activity. An Android activity is the application equivalent to a webpage on a website. Activities are viewed as one screen in an application[19] and are intended to complete one specific task. If another task needs completing then the application should call another activity. Activities can contain fragments which can be thought of as "a modular section of an activity ... (sort of like a 'sub activity' that you can reuse in different activities)"[20]. The modules, submodules, and difficulty nodes in the control flow graph displayed above are activities which contain only list fragments. The list fragment class is a subclass of the fragment class but is adapted to display a list of items, and implement a listener which will act when the user clicks on an item from the list. In the case of the modules node, the list displays all six modules which are examined in Edexcel GCSE mathematics. When the user clicks a specific module, the modules activity collects the information of the module selected, and starts the submodules activity and passes the information of the module selected onto the new activity. The submodules activity then displays the submodules for the module selected by the user. \par

The control flow for this application is minimalist. It only includes pages which are absolutely necessary. The graph for the control flow of the application is intentionally acyclic. Without any loops there is only two ways in which a user can end up at a certain page, they can go forward from the root node, or they can go backwards from a further node. This is simple, and should not confuse users, as they can only follow linear paths within the application. If a user ends up at a page that they have encountered before they must have gone backwards somewhere. \par

\subsection{Data Flow}

Figure \ref{figure:applicationDataFlow} is a graph of the flow of data throughout the application. When an activity starts another activity, it can pass information to it. This enables a single directional information flow throughout the application. That is why the diagram displayed as figure \ref{figure:applicationDataFlow} is a directed graph. The direction of the data is denoted by an arrow, the side of the edge without the arrow is the sender of the data, and the side of the edge with an arrow is the data receiver. 

\begin{figure}[H]
	\centering
	\frame{\includegraphics[width=0.9\linewidth]{./data/applicationDataFlow.pdf}}
	\caption{Diagram representing how data flows through the application.}
	\label{figure:applicationDataFlow}
\end{figure}

The data flow in this graph follows the minimalist design of this application. Data moves along a path in one direction so there is no confusion. A user can go back to a previous activity by using the inbuilt back button in Android. By doing this, the information selected further up the graph (closer to the root node) is preserved, but the information gathered along the edge which the application is now travelling back from is lost. This allows a user to utilise the back button to change the difficulty of the questions that they are answering while keeping the module and submodule information preserved. This enables simple use, where a user does not need to reselect all information to change one specific piece of data. \par

Information is passed along the modules path until it gets to the questions activity. The questions activity then uses all of the information to select the questions that it would display. \par

\subsection{Visual Design}

The visual design of this application is minimalist, in keeping with the rest of the app. Every page only has the necessary things displayed on it, and where applicable, only one thing displayed. Below are a few examples of this minimalist design. Figure \ref{figure:applicationStartPage} is a screenshot of the application's start page taken from the physical Google Pixel 2 device.

\begin{figure}[H]
	\centering
	\frame{\includegraphics[width=0.5\linewidth]{./data/applicationStartPage.png}}
	\caption{Start page of the application screenshot from the physical Pixel 2 testing device.}
	\label{figure:applicationStartPage}
\end{figure}

This screen is made up of the least amount of distinct elements that is possible. As it is the start page of the application it has to have the application name displayed, which is the Maths Help text at the top of the screen. The bottom of the screen is dominated by two buttons, one which starts the modules activity (the modules button), and one which starts the information activity (the app information button). There is nothing else visible on the page, and this is the least amount of elements with which to follow the control and data flow graphs displayed above. \par

An example of the list fragment pages is the modules activity page. This is displayed below as figure \ref{figure:applicationModulesPage}, also taken from the physical Pixel 2 testing device. 

\begin{figure}[H]
	\centering
	\frame{\includegraphics[width=0.5\linewidth]{./data/applicationModulesPage.png}}
	\caption{The modules activity page.}
	\label{figure:applicationModulesPage}
\end{figure}

There is nothing on the page but the selections that the user can choose. This basic design ensures that only what is essential is included on the page. Each of the screens which contains a list could include an explanation at the top informing the user to select one of the options, but this was not included as each user involved in black box testing intuitively understood how the lists worked. It was deemed that an explanation would clutter the page, and could be confusing to users who could see it as condescending. \par

The question display activity also obeys the minimalist design of this application, only containing what is essential for the question to be answered. It can be seen as figure \ref{figure:applicationQuestionPage}, which is a screenshot of the page on the physical Pixel 2 testing device.

\begin{figure}[H]
	\centering
	\frame{\includegraphics[width=0.5\linewidth]{./data/applicationQuestionPage.png}}
	\caption{The question display page.}
	\label{figure:applicationQuestionPage}
\end{figure}

This page consists of the question, an answer box for the user to input text, and a submit button. In the case in which an image is displayed it appears between the question text and the answer box. There is nothing on the page which could confuse a new user. The minimalist design is followed throughout the application, include only what is necessary as to not over complicate the application. \par

\section{Implementation}
\label{section:implementation}

This section will discuss how the project was actually created, providing examples from code where appropriate. It will explain in depth some key features of Android development and how they have been applied to this project, and will also justify key decisions made throughout the development process. 

\subsection{Fragments}

Android activities were explained earlier in the design section of this report. A fragment is very similar to an activity, and has a very similar life cycle. Fragments can be viewed as a modular section of an activity. One activity can contain multiple fragments, and the same fragment can be used in many activities. A fragment is designed to be used when a developer wishes to run the same code in multiple activities, and is a way of preventing developers from rewriting code. As fragments can be added to activities in the same way a text box is, it is simple and quick to reuse code through fragments. \par

Visual elements of the application are stored as XML data and then rendered on the device. Each activity has its own XML data to tell the device how it should be rendered. This allows each activity to have its own unique layout which is created to achieve the goal for that activity. Fragments also have their own XML data which enables a unique layout to suit their specific goals, and they can have their own code. Fragments, however, have to be attached to a parent activity. This allows fragments to act similarly to an activity would in a given situation. \par

In this project fragments have been used in place of activities for lists. This allows for different layouts for different sizes of devices. Although changing the look of the application depending on the size of the device it is running on is currently unused as the application is simple, it could allow for different layouts between tablets and smartphones when the application gets more complicated. It could also allow lists to be displayed next to each other on larger devices. For example, the submodules list could appear next to the modules list when a module is selected. This only works if there is enough space available on the screen to display two lists next to each other. \par

As already stated, this project has used listFragment instead of listActivity to control the modules, submodules, and difficulties lists. There are examples however of where activities have been used as opposed to fragments. The questions page of the application can be seen as an example of this as it is currently an activity where it could have been an activity composed of fragments. This was due to ease of creation as opposed to design choice. However, the displayQuestion activity could be modified to utilise several fragments instead of as an activity. For example, displaying an image could be implemented as a fragment which displays the image, and then the fragment could be included in the displayQuestion activity. It was also created as an activity as the code would not be reused in another place in the application. \par

\subsection{Questions}

Implementing the questions for the application, which were stated as the main feature of the application, consisted of several decisions which were made. Three key decisions were made in regards to how questions were handled by the application with varying consequences. \par

The first key decision that was made regarding the questions in this application involved how the questions were added into the application. It was essential that the questions generated by this application were of GCSE standard so that students could effectively revise for their exams. The decision was made to pull the questions included within the application from past Edexcel GCSE Mathematics papers. Examination of the new Edexcel specification began in the summer of 2017 and was conducted twice an academic year in November and June. This project was conducted between the fall of 2019 and the summer of 2020. This allowed for six different sets of past papers for the current Edexcel GCSE Mathematics specification at the beginning of this project. Each of these sets included three foundation tier papers and three higher tier papers. Each question paper included between 20 and 30 questions, which provides a lower bound for 720 past paper questions provided by Edexcel. This allowed for a large pool of questions to add into the application, spread across all of the modules and submodules tested within the Edexcel GCSE Mathematics specification. \par

The process of adding new questions into the application consisted of labelling each question from a past paper with the module and submodule that it belonged to. These questions, combined with the difficulty of the paper that it was included in, meant that the path that a user would have to take through the application to get to this question could be mapped. It also highlighted the amount of questions which were similar to the one in the exam paper which are already present within the application. This aided to prevent repeat questions, as there were similar questions for the same paper in different past paper sets. It also gave the ability to balance the questions, ensuring that each module and submodule had the same amount, which meant that each topic could be revised thoroughly. \par

The second key decision that was made regarding the questions in this application was their storage medium. There were three key methods of storing the questions for this application, and the two other storage methods are explained in section \ref{section:futureWork}. The three options were: hard-coding the questions into the application, reading the questions from a JSON file on the user's device, or selecting specific questions from the SQLite database included in Android. For this project the questions were hard-coded into the application. This decision was made due to the methodology used to develop this software. As discussed in section \ref{section:methodology} when developing the application an MVP was achieved quickly, with that MVP then being built on to add more features. Because of this, when the question features were being implemented into the application only one question was implemented. Once that one question was successfully tested other questions were added into the application. The fastest way to implement the questions into the application were to hard-code them into a class. Once the initial question was implemented questions were added in batches of four. \par

The third key decision regarding questions was how to hard-code them into the application. A questions class was created which included all of the information that a question would require. A screenshot of the question class can be seen below as figure \ref{figure:questionClass}. 

\begin{figure}[H]
	\centering
	\frame{\includegraphics[width=0.9\linewidth]{./data/questionClass.jpg}}
	\caption{Source code for the question class.}
	\label{figure:questionClass}
\end{figure}

This class includes both the moduleID and submoduleID of the question. It also includes an array list which stores data in string formar. This array list contains the question text data, which is then displayed by the displayQuestion activity. The displayQuestion activity iterates through the array list displaying each individual element in order to create the whole question text. This could have just been stored as one long string, but it is stored as an array list for convenience. It is convenient because it enables the question text to be split on the numerical values which are used within the questions. The final array list will be in the form "text (number text)*" where this form is viewed as a regular expression. This enables the ability to put numerical values into the question text but also to access these values if needed. \par

The string pictureName included in the question class in figure \ref{figure:questionClass} is the file name of the image which will be displayed along with the question text. Currently, only one image can be displayed, although this could be extended to multiple images by storing the images in an array list which would then be iterated through. If this variable is set to the string "None" then the image display view is hidden from the screen, and no image is displayed. The question difficulty is also stored as a numerical value being either zero or one. This field storing a zero indicates that the question is a foundation tier question, whereas a one value being stored indicates that the question is from a higher tier paper. \par

The question's answer is also stored in the question class. All question answers have been rounded to integer precision and all questions include information stating that the user's answer should be rounded to the closest whole number value. On top of the text being included in each question users are prevented from entering non-integer values into the text box as it will only accept whole number values. Currently, no questions which require text based answers are included in the application, although they could be implemented. These kinds of questions would be included in multiple choice format, with radio buttons displaying prospective answers, and users would select the answer that they thought was correct. The application would send the answer as a value between zero and the number of prospective answers displayed to be marked. The background of the correct answer could then be displayed as green to indicate that it is the correct answer. \par

Where a numerical value is used in a question it has been produced in one of two ways. It has either been produced using the java random method, or it is calculable from other numerical values which have been randomly produced for the specific question. This ensures that numerical values for questions are values which have not been seen before by a user, or are not repeated often. This aims to prevent answer memorisation. Answer memorisation is where a user enters an incorrect answer for specific question values, and then remembers the correct answer which gets displayed, and so next time a student is presented with the same question with the same values, they input the answer whihc they remember from the last time they attempted the question and it is marked as correct. This makes it look like the student has learned the method for answering the question, when in reality they have just memorised the answer for those specific values. This is why when a value is used in a question it is generated using the java random method over a specific range of values. This allows the student to see new values often whilst attempting a specific question, but it also allows the answers to be calculated without the use of a calculator. \par

For a specific submodule several questions are generated. When a user selects a specific submodule, one of the questions associated with that submodule is produced to be answered. These questions each have an equal probability currently of being selected, but the frequency could be adjusted to emulate the likelihood of those types of questions appearing in the exams. \par

Figure \ref{figure:testQuestionGeneration} is an extract from the source code of the application. This section of source code generates the test questions and answers, and returns a question class with all of the relevant information in it. 

\begin{figure}[H]
	\centering
	\frame{\includegraphics[width=0.9\linewidth]{./data/testQuestionGeneration.jpg}}
	\caption{Source code for the random question generation}
	\label{figure:testQuestionGeneration}
\end{figure}

This testQuestionGeneration class which is displayed above is a factory class which returns specific questions given the module and submodule details. The calculationFoundationQuestion() method is called when the user selects the number module and the structures and calculation submodule. It returns one of three questions written for the structures and calculation submodule, selected using Java's random number generator. The random question is generated by the appropriate private method in the factory class. For example, calculationFoundationQuestionOne() is a private method which generates a question object with the appropriate information to return. This method generates five random numbers between 0 and 500, and the user is asked to input the smallest number of the five. This question object is then returned to the calculationFoundationQuestion() method which then returns the question object to the displayQuestion activity which then displays the question for the user. \par

\subsection{Control and Data Flow}

A decision was made when it came to the control flow of the application. As is stated above, currently a list fragment is used to display both the modules and submodules lists, as well as the difficulty list. It was decided that the outcome of the modules list fragment was to feed into one submodules list fragment that would collect its list to display based on what module was selected. Instead six list fragments could have been created, one for each set of submodules for each module, or six list activities. This would have greatly increased the complexity of the application from a development perspective, although would not have changed the user experience whilst using the application. Figure \ref{figure:controlFlowComparisons} below shows the control flow diagram for this application, as well as the complex control flow diagram which splits the submodules activity into six individual submodule activities, each representing a separate module.

\begin{figure}[H]
	\centering
	\begin{subfigure}{0.45\textwidth}
		\frame{\includegraphics[width=0.9\linewidth]{./data/applicationControlFlow.png}}
		\caption{A diagram of the control flow of the final application.}
		\label{subfigure:applicationControlFlow}
	\end{subfigure}
	\hfill
	\begin{subfigure}{0.45\textwidth}
		\frame{\includegraphics[width=0.9\linewidth]{./data/adaptedControlFlow.png}}
		\caption{A diagram of the control flow for the hypothetical complex application.}
		\label{subfigure:adaptedControlFlow}
	\end{subfigure}
	\caption{A comparison between the two choices in control flow for the application.}
	\label{figure:controlFlowComparisons}
\end{figure}

The second image, sub-figure \ref{subfigure:adaptedControlFlow}, has almost twice as many activities as sub-figure \ref{subfigure:applicationControlFlow}, and as such would have increased the time spent programming the submodule lists without providing benefit to the users. This is why the decision was made to have one activity which would fetch the appropriate list of submodules depending on which module was picked. This would also make adding a module or submodule simple, as it would just require adding the name of the new module or submodule to the appropriate array. If the other method of fetching a list of submodules was implemented, as sub-figure \ref{subfigure:adaptedControlFlow} displays, then adding another module or submodule would involve changing the control flow of the application by adding extra activities. This would have increased development time and increased the difficulty of maintaining the application if the GCSE Mathematics specification were to change.


\section{Testing}
\label{section:testing}

The project specification[x] stated that this project would implement test driven development. The progress report[x] described that test driven development would not be used as the testing methodology for this project, and that user testing would be used throughout this project in its place. \par

The decision to change from unit testing to user testing was made because this project creates a tangible final product in the form of an Android application. Writing unit tests for an android application was deemed to an inefficient use of time, as this time could be spent improving the user experience of the application. An android application is judged by users for unquantifiable reasons, such as the "feel" of the application. This can only be tested by a person using the application, and since this determines how satisfied a user is with the application testing for the unquantifiable qualities of an application was prioritised over unit testing. \par

\subsection{Testing Methodology}

All application testing was conducted on a physical Google Pixel 2 device, as well as at least one emulated virtual device. The emulation software used is the software built into the Android Studio IDE. The emulator is called Android Virtual Device and [21]"The emulator provides almost all of the capabilities of a real Android device". The emulator provides an environment to test the application on a variety of devices with different capabilities to ensure that the application works on a wide range of Android devices. The application was tested on three virtual devices: 

\begin{itemize}
	\item HTC Nexus One (Smartphone)
	\item HTC Nexus 10 (Tablet)
	\item Google Pixel 3 (Smartphone)
\end{itemize}

These devices give a good variety in Android devices, between both tablet and smartphone, and old and new. These devices were chosen because if the application works well on all four devices, the physical Google Pixel 2 as well as the three virtual devices, then the application was likely to work on most Android devices. \par

Two types of user testing were conducted throughout this project, white box testing and black box testing. \par 

\subsubsection{White Box Testing}

White box testing was conducted after each sprint cycle, where the developer would download the application onto their physical pixel 2 testing device as well as the range of virtual devices mentioned above, and then test that the application compiled and ran without errors. Other small tests were also conducted when testing new features. An example of this feature test would consist of the developer creating expected states of the application given specific inputs, and then the developer would follow the input sequence on all four testing devices, and ensure that the application ended up in the expected state for each device. Visual tests were also conducted, where an activity's display (which can be imagined as the application equivalent of a webpage in a website) in the application was drawn on paper, and then loaded up in the application for each of the testing devices, and the layout on the testing devices was compared with the layout drawn on paper. This testing method was used to find visual errors in the application and was conducted when a visual change was made. \par

\subsubsection{Black Box Testing}

Black box testing was conducted using users outside of the development team. These users were not shown the source code for the application, and this testing was conducted primarily on the physical Pixel 2 testing device. The user would be provided with the application and their use of it would be observed. Before any testing was conducted verbal consent was gained from the user for testing the application through observing their use case. If it was the first time the user had seen the application then they were asked to try to use it as if they had just downloaded it from their normal app store. Whereas, if a user was familiar with the application then they were asked to complete a task which utilised a new feature. They could also be asked to compare visual styles with an image of both the current and previous styles, although the user was not made aware of which style was the current application style and which was the previous application style. Where there was a change in a feature that the user had used before, they were not notified of the change, and how they dealt with the change was noted. \par

Black box testing was conducted after every major feature was added to the application, with the results being used to improve the feature. Users of different mathematical ability were asked to test the application, from colleagues who use mathematics every day in their fields of study, to those who last encountered it when they took their GCSEs. White box testing was conducted throughout development. This was because compilation tests were quick and easy to conduct, taking around one minute per test and requiring access to the physical testing device, and would display errors in the code which would then be fixed before attempting the compilation test again. \par

\section{Risks}
\label{section:risks}

This section will discuss some of the risks associated with this project which would have caused issues if they had occurred. The risks will be outlined, which will explain the risk and how it would have effected the project. Once the risk has been explained then the discussion will be around how the risk was handled in this project. Each method of handling risk will be one of the following four. The risk was either: avoided, transferred, mitigated, or accepted. When a risk is avoided, the cause of the risk is eliminated from the project. A transferred risk is one where the risk was handled by a third party, for example in the case of authentication, if single sign on is used the risk of password leakage is transferred to the third party responsible for the authentication. If a risk is mitigated then the effects of the risks are reduced either by reducing the probability of the risk occurring, or by reducing the impact of the risk on the project. An accepted risk is one where no action is taken before the risk occurs, and if the risk does occur, it is then promptly dealt with without a large impact on the project. Finally, the outcome of the risk is discussed. \par

\subsection{Scope Creep}

Scope creep occurs when the scope of the project slowly increases over time. This usually occurs because of one of two reasons, either a customer keeps on changing their description of the final product, or the project manager adds unnecessary features to the project. As there is no customer directly involved in the development of this project the only way scope creep can occur is if the project manager adds to the initial feature set. \par

Scope creep can be incredibly damaging to a project in one of two ways. The first way is that if the project does not have a predefined deadline and if features continue to be added and changed throughout the development of the project then the project may never be completed. If the project is evaluated for resources at the beginning of the project and scope creep occurs, then there may not be enough resources to complete the project. The second way that scope creep can be detrimental to a project is if that project has a deadline, then adding and changing features means less time is spent on the key features of the project. When a resource is constrained scope creep can waste the resource and leaves less for important features. \par

For this application scope creep would be changing or adding features which cause current progress to be scrapped. This is because time is the key limiting resource for this project. Any time wasted in this project limits the amount of requirements which can be completed by the end of the project. \par

Scope creep was handled by outlining the features necessary for this application in the form of requirements in the project specification, as well as defining a successful project. The requirements in the project specification outline the features which should be worked on throughout this project and the definition of success provided a baseline for the features to be provided in the project. It also gives a rough idea for the order that they should be completed in, specifically for dependent requirements. This mitigates scope creep by lowering the probability that it occurs. This is due to requirements being laid out in the beginning of the project, with only minor changes being allowed throughout the project. So as long as the requirements are completed before any other feature is attempted to be added to the project then it will not creep. \par

This project did not suffer from scope creep. The requirements were a key reason for this as they were converted to user stories which then dictated what was completed in a specific sprint cycle. As work was only conducted on user stories there was no ability for features which were not included in the initial requirements to be worked on, and so no time was wasted. \par

\subsection{Personal Issues}

Due to the development team containing one person who was irreplaceable any personal issues on behalf of that individual could affect the project. Personal issues could include a high workload, family issues, or a sudden pandemic. Any issues which would prevent the individual from working on the project due to reasons outside of the development cycle are classed under personal issues. \par

Personal issues are things which cannot be prevented as they are unpredictable and can have varying affects on a project. That means that personal issues have to be accepted. As personal issues are a group of issues which are unrelated to the project, but affect it, it is possible to mitigate or transfer these individual issues. For example, if a personal issue is that the only member of the development team were to fall ill and be unable to work for two weeks, then this can be mitigated with a deadline extension, so although personal issues as a group have to be accepted, individual issues are handled appropriately. \par

Accepting the personal issues with this project did not have a major effect on the outcome. Personal issues did impact the pace of development in the first term of this project, as the development team had other pressing priorities to attend to. This, alongside poor project management, caused the project to fall behind schedule. This time was made up due to good project management throughout the second term of this project, and so the project was completed on time. \par

\subsection{Poor Project Management}

Poor project management is caused by a project manager who is not performing as well as they should. It appears in many forms, one of which is project creep which is discussed earlier. It could also show as poor productivity, or disagreements within a development team. It can severely affect a project. Poor management can cause the project to fail, whereas good management can save a project. \par

The three main causes of poor project management are:

\begin{itemize}
	\item an inexperience project manager
	\item poor communication within a development team
	\item lack of sufficient planning.
\end{itemize}

Each of these issues alone can slow the development process and can cause issues affecting the final product. For effective project management at least the three main causes of poor project management have to be handled appropriately. \par

Initially throughout the first term and the Christmas break, project management was not handled appropriately. This, alongside personal issues discussed earlier, caused the project to be delayed by four weeks at the beginning of the second academic term. Poor project management was initially accepted, although as risks are both monitored and controlled on an ongoing basis throughout the project to ensure that the project will be successfully completed. This monitoring highlighted that if the current project management techniques were allowed to continue then the project would be a failure. Poor project management was then changed from being accepted to being mitigated. \par

The plan that was implemented to fix the poor project management from the first half of the development cycle tackled two of the three main causes of project management which affected this project. Poor communication within a development team cannot be a problem if the development team only consists of one member, so that was not an issue with this project. An inexperienced project manager was fixed through the project leader taking CS352 Project Management for Computer Scientists. This taught the project manager how to handle time management through the use of calculating the critical path for this project. This highlighted the requirements which would delay the project's completion if it was delayed. This provided the knowledge of the order that the requirements should be completed in to minimise the amount of time spent waiting for another feature to be completed. This saved time throughout the development process which enabled the completion of more requirements and ultimately lead to the success of the project. \par 

On top of the project manager gaining a greater understanding on how to effectively manage a software development project, there was an increase in planning throughout the second term of the academic year. The change which had the biggest impact on the project was giving the project specific development time. The first term consisted of working on the project whenever there was spare time. This unpredictability meant that different amounts of hours were put into the project each week. This inconsistency aided in the project falling behind, as less work was completed in the first term because the amount of hours that were predicted to be worked by the initial schedule was not being completed. Giving the project specific times each week to be worked on increased the work rate. All of Wednesday was dedicated to project work throughout the second term, and this enabled productivity to be greatly increased throughout the second term. \par

Eliminating the two main causes of poor project management enabled the project to be completed on time to a good standard. \par

\section{Project Management}
\label{section:projectManagement}

This project was a large software engineering project, especially for a team size of one. This meant that effective project management was required to keep the project going and make it a success. 

\subsection{Schedule}

A schedule was created early, and updated throughout the project to represent and handle the amount of work which had been completed. Two key versions of the schedule were displayed to various stakeholders throughout the project in key documents. These key documents were the project specification and the progress report. 

\subsubsection{Project Specification Schedule}

At the beginning of this project a schedule was created. This schedule aimed to complete all of the requirements stated in the project specification[x] by the end of the project. The schedule was created using a technique common in industry which involved some numbered cards. Each requirement was reviewed with each review consisting of analysing the relative difficulty of the requirement with respect to all the other requirements. The difficulty was quantified using the numbered cards, the cards would follow a logarithmic pattern, doubling with difficulty each time. The lowest valued card was \textsuperscript{1}/\textsubscript{4} and the largest value was 16. Any requirement with a relative difficulty value greater than 16 was then broken down into sub-requirements and these sub-requirements would then be reviewed, and broken down further if their relative value was also above 16. This is called planning poker in industry. \par

This relative difficulty in requirements was used to create a rough schedule for the project. The schedule created for the project specification can be seen as a Gantt chart in figure \ref{figure:projectSpecGanttChart}. \par

\begin{figure}[H]
	\centering
	\frame{\includegraphics[width=\linewidth]{./data/projectSpecGanttChart.png}}
	\caption{Gantt chart of the schedule produced for the project specification.}
	\label{figure:projectSpecGanttChart}
\end{figure}

This schedule was an optimistic schedule. It assumed all of the requirements would be completed by the end of the project, whereas the project specification's definition of success did not need all requirements to be completed. 

\subsubsection{Progress Report Schedule}

The progress report submitted on 24\textsuperscript{th} November 2019 updated this schedule, once it was realised that development was slower than initially intended through term one of this academic year (October 2019 - December 2019) due to personal reasons. The new schedule, referred to from now on as the Christmas schedule, again made the assumption that every requirement would be completed before the deadline. To mitigate against the slow development conducted through term one, the Christmas schedule increased the workload of the project throughout the Christmas period (8\textsuperscript{th} December 2019 - 8\textsuperscript{th} January 2020) and throughout term two of the project. \par

A Gantt chart created for this updated schedule can be viewed below as figure \ref{figure:progressReportGanttChart}

\begin{figure}[H]
	\centering
	\frame{\includegraphics[width=\linewidth]{./data/progressReportGanttChart.png}}
	\caption{Gantt chart of the updated schedule produced for the project specification.}
	\label{figure:progressReportGanttChart}
\end{figure}

This Gantt chart represents everything that was dated before 24\textsuperscript{th} November 2019 is correct as the work was already completed before the progress report was written. Anything after this date is an estimate, and comparisons for future work should be compared to this. This schedule is also, like the project specification schedule, optimistic. 

\subsubsection{Schedule Evaluation}

The image displayed below as figure \ref{figure:finalSchedule} shows the final schedule at the end of the project. 

\begin{figure}[H]
	\centering
	\frame{\includegraphics[width=0.9\linewidth]{./data/finalScheduleColour.jpg}}
	\caption{The final schedule created at the end of the project.}
	\label{figure:finalSchedule}
\end{figure} 

This figure shows information about each requirement, as well as some key deliverables produced throughout the project. It shows the start and end dates of the requirements, as well as the duration of each requirement, in days. The requirements are colour coded. Requirements with a green background have been fully completed at the end of the project. A yellow background for the requirement represents a started but uncompleted requirement, and a red requirement was not started during the project. \par

This final schedule was behind both the project specification schedule and the progress report schedule. The reasons for this are discussed in the risks section of this report. All of the schedules are optimistic and assumed that all of the requirements would be met. This was different to how the definition of success for this project was defined. Even though the project was behind the optimistic schedules work was still completed at a satisfactory rate, and so even though the schedule was not followed to the letter it was used as guidance for how long each requirement should take. Although, if time was not lost during the first half of development due to the risks discussed earlier in this report then these optimistic schedules could have been met. \par

\subsection{Tools}

This project utilized several tools to aid in the development of this Android application. Two of the main tools used to add value to this project are outlined below, one from a development point of view and one from a project management standpoint. 

\subsubsection{Android Studio}

Android Studio[22] was used as an IDE (Integrated Development Environment) throughout development as it comes with many useful built in features. These features are specifically designed with android app development in mind. Examples of these features are: 

\begin{itemize}
	\item Rendering engine which converts XML into how it will appear in the application itself
	\item Android phone emulator
\end{itemize}

The rendering engine enables both the ability to edit the XML tags and see the effects in real time, as well as the ability to edit directly into the rendering engine with a drag and drop feature. This enables quick prototyping for new pages through the use of the drag and drop feature whilst also allowing a much larger degree of control through the ability to directly edit the XML. \par

Use of the android phone emulator enabled quick and efficient testing cycles. When a feature was initially developed, it could be tested on a wide range of devices through the phone emulator. This add on enabled the emulation of any modern phone through the ability to download files specific to that device. The app could then be tested on an accurate test bed, seeing how the app would look and work on a mobile device. Tablets could also be emulated, which helped with ensuring the application worked on a wide variety of devices. \par

The Android Studio IDE greatly increased the efficiency of development. This was encouraged through the notion that it was only designed specifically for android development, and so could focus entirely on that. Being built upon the IntelliJ platform meant that code editing was smooth and intuitive. \par

\subsubsection{Git}

Git was used as version control for this project. This enabled access to the project from any device, anywhere, anytime. It also enabled the ability to revert back to an earlier version of the project if something went wrong. This meant that development could take place without fear of failure, as if anything went wrong the project could just be reverted to the last working state. \par

The repository for this project can be found at the link in reference [23]. This repository is public and can be accessed by anyone. This enables anyone to see the code that would be running on their machines. \par

Git also logs the time when a change is made to the code. This enabled precise tracking of requirement completion and aided project management. Git commit messages for this project were all deliberately helpful. Git commit messages were written so that they would explain the changes made to the codebase, so they could be logs of when project requirements were completed. The git commit messages for this project can be viewed and give a detailed timeline of when requirements were completed and features were added. This was incredibly useful as it gave updates of any changes made to the codebase as soon as it was available to anyone who follows the repository, and also was used to create a timeline of when requirements were completed for this final report. \par

\subsection{Miscellaneous Project Management}

Weekly meetings were held with the project supervisor to keep key stakeholders in the project up to date with development. Alongside weekly meetings, access was given to the project supervisor to the  github repository early in development. This enabled the project supervisor to directly monitor completed work. \par

A document in the github repository called notes.txt is a file which was edited after every development session was completed. This file contained notes on what was intended to be completed during each development session, as well as what was actually completed. A substantial amount of time was spent at the beginning of early development sessions on remembering what had been completed throughout the previous development session, what state the code was in, and what needed to be completed in this development session. This file cut down the time spent at the beginning of development sessions looking through previous git commits, and so overall this file caused development to be more efficient. \par

The first term of this project took a very relaxed stance on project development which meant that any free time within a working week was dedicated to this project. For the first five weeks of the first academic term this was fine, but it meant that as the academic year progressed, less time was spent on the project per week. This inconsistency in development time made it progressively more difficult to follow scrum sprint cycles. This inconsistency was incredibly unproductive and if development continued at this pace then the project would have been deemed a failure, as is discussed in section \ref{section:risks} about risks. \par

\section{Results}
\label{section:results}

This section will evaluate the final application produced by this project and compare it against the requirements and definition of success set out at the beginning of this report. It will also contain the analysis of the project by the author. 

\subsection{Requirements}

The final application satisfies all of the must requirements. This means that the final application can display a question for users to answer, have an input field accepting numerical values, and when an answer is submitted, compare it to the pre-calculated answer and display whether the two answers match. If the two answers are different, then the correct answer is displayed for the user to see. \par

One of the three should requirements for this project was completed outright. Requirement 7 was changed between the progress report and the project presentation. It was changed from "Have a variety of testing difficulties, ranging from easy to hard for each module" to "Have two separate testing difficulties, foundation and higher, for each module". This change was made as the Edexcel GCSE Mathematics exams are split into two categories, of foundation and higher. The foundation tier of exams range from grades 1-5 and the higher range from grades 4-9, with the higher tier covering all content in the foundation tier and build on it. As this tier system already existed, and changing to easy, medium, and hard would only confuse students, the decision to switch to Edexcel's system was made. \par

Requirements 8 and 9 were not completed but are in progress. These requirements have been started and prototypes for each requirements have been made. Model working out has been created for three different questions in the algebra topic. These model working out are displayed when the user submits an incorrect answer. \par

The could requirement has not been completed. This is because requirements 8 and 9 have not been completed so work for requirement 10 has not been started. \par

As non-functional requirements are difficult to test, it is difficult to determine in absolutes whether these requirements have been met. Any tests conducted with 12 year olds to determine their comprehension of the questions in the application would not have been representative of the population as a whole, this method was not used to attempt to assess the completion of the non-functional requirements. Instead, user testing was conducted with colleagues who work with students of GCSE age range. Their feedback was incorporated into the development process as explained in the testing section of this report. Their feedback on the final application was that the functional requirements have been completed for almost all students who are in the English year 11, and the majority of students in the English year 9. Because of this, these functional requirements have been deemed completed for this project. \par

\subsection{Results Overview}

By the definition of success laid out in the project specification, as well as discussed in the introduction to this report, this project was successful. All of the must requirements were completed, as well as one of the three should requirements, although the could requirement was incomplete. \par

Overall, the project produced a proof of concept that the MyMaths homework functionality could be turned into an application, as was the end goal. This project could be released on the Google play store as it is, although if this project were to continue the application would not be uploaded to the Google play store for another two to three months of full time work. This extra time would be spent completing the should requirements, and working on the could requirement for content which is examined on the foundational exams. \par

This project could have been improved with better project management throughout the first term of development. The project fell behind early in development due to personal issues on behalf of the author. Alongside said personal issues, time management was poor throughout the early stages of this project, and project management techniques which were key in improving the project in term two were not used. Improvement was shown in project management the further into the project development got. Improvements were made to the authors time management skills, such as dedicating specific time each week to work on the project. These improvements had a large impact on the final product, and counteracted the previous loss of development time from the first term. \par

Advanced software engineering techniques have been included in this project, for example the use of a factory class for generating test questions. These were included where appropriate. \par

This project was worthwhile as it highlights that more work can be completed on mobile GCSE revision applications, whilst also displaying the amount of work required to build an app of this kind. Students would benefit from an application, endorsed by teachers, which aided in a student's education in GCSE mathematics, whether that application is similar to one built by this project, or another application with slightly different goals like Khan Academy. Reducing the barrier of entry for students to be able to improve their education only requiring access to a smartphone could increase GCSE grades, as most students have regular access to a smartphone, as referenced in the introduction of this report. \par

\section{Conclusion}
\label{section:conclusion}

To conclude, this project was a success. It is a proof of concept which highlights a gap in the GCSE Mathematics revisions market. The definition of project success laid out in the project specification document for this project. On top of that the final application produced by this project could be used by GCSE students to aid their revision of GCSE Mathematics. \par

A GCSE Mathematics revision application endorsed by schools will most likely be produced in the future. Throughout the development of this project MyMaths have released an application which is accessible through tablets which currently has a small amount of content accessible, with more being added throughout the coming months[30]. MyMaths have created this tablet application as all of their content before November 2019 was written in the flash multimedia software platform. Adobe announced in July 2017 that it was going to "stop updating and distributing the Flash Player at the end of 2020 and encourages content creators to migrate any existing Flash content to new open formats"[31]. This means that MyMaths is migrating all of their content away from flash, and in doing such are producing an application available on tablets. \par

This project produced software which could be released as an application onto the Google Play store and would be useful for students looking to sit their GCSE Mathematics examinations from November 2021 onwards. All of the questions in the application have currently been taken from non-calculator past papers and so no calculator is necessary. Although, the questions could be adapted to include more difficult numbers which would require a calculator. This application could also be updated with new questions after each exam cycle is complete. \par

The development of this application could continue into a commercial product. If the improvements in the future work section of this report is completed then the application could be brought up to paid quality, and so could be released commercially. It could also be continuously updated with new features, and the scope of the project increased. \par

\section{Future Work}
\label{section:futureWork}

This project was intended as a proof of concept, trying to create an android application which mimics the homework feature from MyMaths. The final product from this project is not perfect, and could be extended and released to the Google play store. This section of the report will outline key changes which would be made to the project if restarted. After, it will explain extensions which would be made to this project if it were to be continued. \par

\subsection{Project Improvements}

These improvements are the changes that would be made to the application if the project was started again from scratch.

\subsubsection{Question Storage}

In the current implementation of the application questions are hard coded into the application. This has several disadvantages. One of the disadvantages of hard coding the questions into the application is that any changes made to questions would only be available to users through updates to the application. This delays question changes getting to users and would cause lots of small app updates to fix small problems in questions. Another disadvantage to this hard coding question storage is that users cannot add their own questions to the pool of questions in the application. This is because users do not have access to, and cannot change, the source code of the application. \par

There are two methods which could be implemented to store questions in the application, both with benefits and drawbacks. Both methods would require a slight change in the logic of how questions are read and displayed. \par

The first method of question storage would be to write the questions to a JSON file which is stored locally on the system. This would allow the manipulation of questions (adding questions, changing current questions, etc) locally. This would enable users to add questions just by appending an element to the JSON file, which could then be implemented easily. It would only require a small change of the current question displaying logic. This is because the current question display logic is written to take an Array list storing string types and iterate through it until all elements are displayed, this could be converted to iterating through the data in a JSON object until there is no more data. Also, the data would not be rewritten each update. A second JSON file can be added which contains just the user added questions, and these questions can be appended to the questions JSON file with each update. \par

The second method of question storage would be to insert elements into a database. This has the same benefit of being local storage and so adding user questions is as easy as inserting the question information into the database. SQLite is built into Android and so it would be easy to implement. It would, however, require a complete rewriting of the question displaying logic. This is due to the question logic iterating through the Array lists. Instead, the question displaying logic would query the database, fetch some relevant data, and find a way of displaying that data and generate the random numbers necessary as per the requirements of this project. \par

If this project was restarted then the method which would be implemented would depend on the amount of question data. For smaller question amounts the JSON file would be used. This is because iterating through a JSON file until the right question is found would not add too much time onto question generation. The convenience of creating a working implementation of the JSON storage method would be too great to use the database method. Although, when the JSON file gets large, and lots of questions are contained in it, iterating through questions until the correct question is found would be time consuming. In this situation the database method would be used. SQLite would be much faster at locating a specific row in the database than iterating through the JSON file. \par

\subsubsection{Accessibility Improvements}

If this project were to be restarted then accessibility issues would be much higher on the priorities list than it was. Questions could be played in audio form for those users who are blind. Audio files could be saved to the users device and concatenated to display any questions that could be generated from the application. Also, when an answer is inputted that number could be spoken aloud, or in the case of a multiple choice question the selected answer could be read aloud. This would enable a user which has visual impairment to be able to use the application. \par

There could also be a setting which increases the font size of all text in the application, and would also increase image size. This would also aid users with visual impairments, and would be built into the application from the first version. \par

\subsubsection{Content Grouping}

Content could be grouped into appropriate ages as well as modules. This would aid younger students, those ranging from academic years 7 to 9, to utilise the application to revise. This would increase the target audience of the application, and aid more students in their understanding of GCSE Mathematics. \par

Currently, any students in that age range who wish to use the application could get questions above their skill level. This is counterproductive to the aims of this application, as students would be frustrated that they cannot answer a given question as they have not covered the content, and they are less likely to continue to use the application. Instead, with content grouped into appropriate academic years, as well as modules, students can be certain that questions that they are tackling in the application are appropriate to their skill level. \par

\subsubsection{Trello}

Project management was an issue early on in the project. Even throughout the second term of this academic year the project management was better but still limiting the achievement in the project. If the project management had been better then more work would have been completed for the final application. \par

If the project was restarted then some decisions would be changed to improve productivity. The first decision which would be done differently would be to follow some project management techniques. Development would have been started later to allow for effective preparation of a project of this size and magnitude. Certain PMBOK processes would have been followed in the planning phase of this project. For example, the requirements of this project would have been a more defined method for creating a schedule would have been used. Developing a schedule should have consisted of defining activities to be completed which produce deliverables, sequence the activities depending on their dependencies, estimating how long each activity would take, and then creating the timeline for the project. Instead, the time requirements would take was estimated, and estimating the completion of a task is harder than estimating the completion of a deliverable. All of the requirements were written as tasks to be completed and not written as deliverables. \par

Deliverables for this project would be made from the requirements, which as stated above would have made estimating the time to complete each requirement would have been easier. Once the deliverables for this project were created, they could then be entered into a Trello[29] board and more effectively managed. Trello is an organisational tool, and has been used by the author as a to-do list whilst writing this report. Boards can be created which contain deliverables which remind team members what is still to be completed. Figure \ref{figure:trelloBoard} shows the Trello board for this report: 

\begin{figure}[H]
	\centering
	\frame{\includegraphics[width=0.9\linewidth]{./data/trelloBoard.png}}
	\caption{Trello board used as a to-do list for this report.}
	\label{figure:trelloBoard}
\end{figure}

The figure above displays a to-do list for this report at the time of writing this section. Using Trello as a to-do list for the deliverables of this report greatly increased productivity. This was due to increased preparation time for the final report, as opposed to for the development cycle. \par

If this project was restarted from scratch then Trello would be used, alongside some project management techniques to effectively plan the development of the application. \par

\subsection{Project Extensions}

Project extensions are features which would be added to the application if this project continues. 

\subsubsection{Currently Uncompleted Requirements}

If this project continues the first thing that will be completed are the three uncompleted requirements. These are requirement 8, 9, and 10. This will mean that the application will track users recent feedback, display model answers when the user gets an answer incorrect, and there will be some kind of classroom like learning resource. \par

These requirements were in the original scope of the project but due to the project running out of time they were incomplete. This is why they would be completed first if extra time was added onto this project, as requirements 8 and 9 are already in the prototype phase. 

\subsubsection{User Validation}

User validation would be a key feature which would be added to the application. Two types of user accounts would be created, those available for students and those available for teachers. Student accounts will be able to attempt any questions from any module in any year group. They would have full access to all testing and learning content at any time. Teachers will have the ability to create classes which will contain a group of students from the same academic year. \par

User validation was outside of the scope of the current project as there are many legal issues with storing user data. This application would have had to register under the data protection act and ensure that all data collection is compliant under the new GDPR laws as this application would be aimed at being used primarily in the United Kingdom. It was deemed that the potential legal issues which come with storing user data was not worth the risk for a project of this size and scope. This is why if the scope was increased and time was added on then these legal issues will be tackled. \par

A user would need to be able to register an account with an email, whether that is a school specific email or not, and use the application. A single sign on (SSO) feature can also be added to allow users to sign in with other popular accounts such as Facebook and Google accounts. This would shift the burden of sign on security to the validating third party, although would not be practical for all cases. A user may not have a third party account and so there would need to be the ability for a user to register an account with their email as well. \par

\subsubsection{User Added Questions}

A feature which would be useful for users of this application is users being able to add their own questions to the question set in this application. This was deemed outside of the scope for the current project as this is not a feature available by any of the major GCSE Mathematics revision applications. This would be added after the question storage technique would be changed, as currently there is no way to automate the question adding process for a student due to how questions are stored. There is also no way to have user specific questions as the questions are built into the source code for the application. \par

Once the question storage method is changed from the current method used to one of the two other methods discussed above, giving users the ability to add questions would be implemented. This feature would be added only for user initiated testing and would be analysed by the statical tools separately to the questions added into the application by the author, or a student's teacher. This is to prevent students from abusing the question adding process to add questions which have a single, simple answer, or are much simpler than other questions in the same skill level. If the questions were not analysed separately then a student could abuse the question adding mechanism so that they get 100\% in a homework or revision set. \par

This would be implemented by adding an activity which would allow the user to input details for a new question. Users would utilise radio buttons to select the module and submodule for the specific question, as well as the year if that had been implemented when user added questions is. There would be a text box where users could type out the question text, and a specific set of characters would be used to represent variables which the application has to generate. This set of characters would then be searched for using regular expressions. \par

\subsubsection{MyMaths Homework Feature}

Given more time with this project a homework feature, similar to that of MyMaths, could be implemented. This would allow teachers to set classes revision tests to be completed by a given date, and with over a specific percentage score. The students would then be notified of the homework with a push notification, as well as something visual within the application, preferably on the homepage of the application so students do not miss it. \par

Students could then attempt the specific homework with their percentage score being sent back to the teacher to analyse. If a student achieves a score below the passing grade for that homework, then they could reattempt the homework, or look at some revision materials for that specific topic. These homework questions would be separate from the user added questions, for reasons of dishonesty mentioned above. \par

Teachers could also be provided with the ability to select specific questions for a topic, and change the appearance probability. This would grant teachers the flexibility to customise a homework specifically for their class. \par

\section{Legal, Social, Ethical, and Environmental Issues}
\label{section:issues}

This project did not face many legal, social, ethical, or environmental issues as it was a safe topic. There were no environmental or social issues faced for this project. There were limited legal and ethical issues which will be discussed under their respective subheadings. 

\subsection{Legal Issues}

There was one legal issue faced throughout this project. This issue was related to the tools which were used. As the final source code was not all written by the development team for this project; some of the code was generated through the tools used to aid this project, for example any code pre-generated through the Android Studio IDE was included in the final source code. This could create legal issues due to copyright laws. \par

This is why the licensing for all the tools used is linked below as a reference, and relevant sections are quoted to ensure that there are no legal issues with this project. \par

\subsubsection{Android Studio}

Android Studio is built on top of the IntelliJ platform. It shares the same license as IntelliJ, as is shown in this comment on a blog post by the creator of IntelliJ Dmitry Jemerov (Yole in the blog post) [24]. The IntelliJ IDEA[25] (the IntelliJ IDE) is licensed under two licenses, but as is explained in the blog post, the licensed version of IntelliJ IDEA that is used as the basis of Android Studio has the Apache 2 license. The relevant sections of the Apache 2 license are as follows: "You may reproduce and distribute copies of the Work or Derivative Works thereof in any medium, with or without modifications, and in Source or Object form"[26]. This allows for the use of the IntelliJ and Android Studio IDEs for commercial and independent development which this project falls under, and so there are no legal issues with using Android Studio as the IDE for this project. \par

\subsubsection{Git}

The majority of git is released under the GNU public license [27] which enables copying, modifying, or distributing git under any circumstances. The relevant sections of the GNU licence (from the preamble): "By contrast, the GNU General Public License is intended to guarantee your freedom to share and change free software--to make sure the software is free for all its users.". This enables free use of git for this project. \par

Small sections of git are not licensed under the GNU public license, but under the GPL (GNU Lesser Public License) [28]. Relevant sections from the GPL preamble: "By contrast, the GNU General Public Licenses are intended to guarantee your freedom to share and change free software--to make sure the software is free for all its users.". This enables the free use of git for this project. \par

As these are the only two external tools used throughout this project, and the only tools which could have impact on the source code of this project, then there are no legal issues with this project. Any other materials were produced by the author specifically for this project, and so there are no legal problems with using these materials. \par

\subsection{Ethical Issues}

This project only contains one key ethical issue, which is to do with the testing of the final product. Testing was conducted on colleagues who gave verbal and written consent to test the application. User testing was conducted where the user was observed using the application. This testing method makes this project one which does not require ethical review under department guidelines which can be found here [29]. "Student projects with primarily an educational purpose, for example asking peers or family members to test software as part of your course. Such projects should not include any form of deception or coercion or involve vulnerable groups (e.g. schoolchildren)." As testing was not conducted on school children for this project, instead user testing was conducted on colleagues over the age of 18, this project is ethically sound. 

\section{References}
\label{section:references}

[1] - https://www.gov.uk/government/speeches/reformed-gcses-in-english-and-mathematics (Accessed 29\textsuperscript{th} March 2020) \par

[2] - https://qualifications.pearson.com/en/qualifications/edexcel-gcses/mathematics-2015.html (Accessed 29\textsuperscript{th} March 2020) \par

[3] - https://www.pewresearch.org/internet/2018/05/31/teens-social-media-technology-2018/ (Accessed 30\textsuperscript{th} March 2020) \par

[4] - https://www.comscore.com/Insights/Presentations-and-Whitepapers/2017/2017-US-Cross-Platform-Future-in-Focus (Accessed 30\textsuperscript{th} March 2020) \par

[5] - https://gs.statcounter.com/os-market-share/mobile/worldwide (Accessed 30\textsuperscript{th} March 2020) \par

[6] - https://qualifications.pearson.com/content/dam/pdf/GCSE/mathematics/2015/specification-and-sample-assesment/gcse-maths-2015-specification.pdf (Accessed: 6 October 2019) \par

[7] - https://www.mymaths.co.uk/ (Accessed 22\textsuperscript{th} March 2020) \par

[8] - https://www.mymaths.co.uk/help.html (Accessed 22\textsuperscript{th} March 2020) \par

[9] - https://www.khanacademy.org/ (Accessed 22\textsuperscript{th} March 2020) \par

[10] - https://www.khanacademy.org/math/trigonometry/trig-with-general-triangles/law-of-sines/v/law-of-sines (Accessed 22\textsuperscript{th} March 2020) \par

[11] - https://apps.apple.com/gb/app/khan-academy/id469863705 (Accessed 22\textsuperscript{th} March 2020) \par

[12] - https://play.google.com/store/apps/details?id=org.khanacademy.android (Accessed 22\textsuperscript{th} March 2020) \par

[13] - https://mathswatch.co.uk/ (Accessed 25\textsuperscript{th} March 2020) \par

[14] - http://www.mathswatch.co.uk/subscription/4594745491 (Accessed 25\textsuperscript{th} March 2020) \par

[15] - http://coolt.ch/notizen/wp-content/uploads/2016/02/Head-First-Android-Development-2015.pdf (Accessed 31\textsuperscript{th} March 2020) \par

[16] - https://developer.android.com/docs (Accessed 31\textsuperscript{th} March 2020) \par

[17] - https://github.com/AlphaMustang (Accessed 31\textsuperscript{th} March 2020) \par

[18] - https://trello.com/ (Accessed 8\textsuperscript{th} March 2020) \par

[19] - https://developer.android.com/guide/components/activities/intro-activities (Accessed 28\textsuperscript{th} March 2020) \par

[20] - https://developer.android.com/guide/components/fragments (Accessed 28\textsuperscript{th} March 2020) \par

[21] - https://developer.android.com/studio/run/emulator (Accessed 26\textsuperscript{th} March 2020) \par

[22] - https://developer.android.com/studio (Accessed 17\textsuperscript{th} April 2020) \par

[23] - https://github.com/AlphaMustang/thirdYearProject (Accessed 28\textsuperscript{th} March 2020) \par

[24] - https://blog.jetbrains.com/idea/2013/05/intellij-idea-and-android-studio-faq/\#comment-4939 (Accessed 29\textsuperscript{th} March 2020) \par

[25] - https://www.jetbrains.com/idea/features/ (Accessed 29\textsuperscript{th} March 2020) \par

[26] - https://www.apache.org/licenses/LICENSE-2.0 (Accessed 29\textsuperscript{th} March 2020) \par

[27] - https://github.com/git/git/blob/master/COPYING (Accessed 22\textsuperscript{th} March 2020) \par

[28] - https://github.com/git/git/blob/master/LGPL-2.1 (Accessed 22\textsuperscript{th} March 2020) 2020) \par

[29] - https://warwick.ac.uk/fac/sci/dcs/teaching/ethics (Accessed 22\textsuperscript{th} March 2020) \par

[30] - https://support.mymaths.co.uk/technical-support/what-mymaths-content-has-been-updated/ (Accessed 16\textsuperscript{th} April 2020) \par

[31] - https://theblog.adobe.com/adobe-flash-update/ (Accessed 16\textsuperscript{th} April 2020) \par

[32] - Head first android development pg 312.


\end{document}