\documentclass{article}

\usepackage{graphicx,color,tabularx,amsmath,amssymb,fancyvrb,soul}
\usepackage{subcaption}
\usepackage{float}

\begin{document}

\section{Abstract}

\section{Introduction}

\section{Main Body}

\section{Project Management}

\subsection{Tools}

This project utilized several tools to aid in the development of this Android application. Two of the main tools used to add value to this project are outlined below, one from a development standpoint and one from a project management standpoint. 

\subsubsection{Android Studio}

Android Studio was used as an IDE (Integrated Development Environment) throughout development as it comes with many useful built in features. These features are specifically designed with android app development in mind. Examples of these features are: 

\begin{itemize}
	\item Rendering engine which converts XML into how it will appear in the application itself
	\item Android phone emulator
\end{itemize}

The rendering engine enables both the ability to edit the XML tags and see the effects in real time, as well as the ability to edit directly into the rendering engine with a drag and drop feature. This enables quick prototyping for new pages through the use of the drag and drop feature whilst also allowing a much larger degree of control through the ability to directly edit the XML.\\

Use of the android phone emulator enabled quick and efficient testing cycles. When a feature was initially developed, it could be tested on a wide range of devices through the phone emulator. This add on enabled the emulation of any modern phone through the ability to download files specific to that device. The app could then be tested on an accurate test bed, seeing how the app would look and work on a mobile device. Tablets could also be emulated, which helped with ensuring the application worked on a wide variety of devices.\\

The Android Studio IDE greatly increased the efficiency of development. This was encouraged through the notion that it was only designed specifically for android development, and so could focus entirely on that. Being built upon the IntelliJ platform meant that code editing was incredibly smooth and intuitive. \\

\subsubsection{Git}

Git was used as version control for this project.

\section{Results}

\section{Decisions}

\section{Conclusion}

\section{Future Work}

\section{Legal, Social, Ethical, and Environmental Issues}

\end{document}